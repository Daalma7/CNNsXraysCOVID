\documentclass[11pt,a4paper]{article}

% Packages
\usepackage[utf8]{inputenc}
\usepackage[spanish, es-tabla]{babel}
\usepackage{caption}
\usepackage{listings}
\usepackage{adjustbox}
\usepackage{enumitem}
\usepackage{hyperref}
\usepackage{boldline}
\usepackage{amsmath}
\usepackage{amssymb, amsmath}
\usepackage[margin=1in]{geometry}
\usepackage{xlop}
\usepackage{soul}
\usepackage[ruled,vlined,linesnumbered]{algorithm2e}
\usepackage{amsthm} %Paquete para la terminología matemática
\usepackage[ruled,vlined,linesnumbered]{algorithm2e}
\usepackage{subfigure}
\usepackage{listings}
\usepackage{color}
\usepackage{float}
\usepackage{colortbl}
\usepackage[table,xcdraw]{xcolor}


\definecolor{dkgreen}{rgb}{0,0.6,0}
\definecolor{gray}{rgb}{0.5,0.5,0.5}
\definecolor{mauve}{rgb}{0.58,0,0.82}

\lstset{frame=tb,
  language=Python,
  aboveskip=3mm,
  belowskip=3mm,
  numbers=left,
  stepnumber=1,
  showstringspaces=false,
  columns=flexible,
  basicstyle={\small\ttfamily},
  numberstyle=\tiny\color{gray},
  keywordstyle=\color{blue},
  commentstyle=\color{dkgreen},
  stringstyle=\color{mauve},
  breaklines=true,
  breakatwhitespace=true,
  tabsize=3
}
%Entorno de la librería matemática (Macros para que no salga en inglés).
\newtheorem{theorem}{Teorema}[section]
\newtheorem{corollary}{Corolario}[theorem]
\theoremstyle{definition}
\newtheorem{definition}{Definición}[section]

% Meta
% Custom
\providecommand{\abs}[1]{\lvert#1\rvert}
\setlength\parindent{0pt}
\definecolor{Light}{gray}{.90}
\newcommand\ddfrac[2]{\frac{\displaystyle #1}{\displaystyle #2}}
\newcommand\tab[1][1cm]{\hspace*{#1}}

\begin{document}
\begin{titlepage}
  \centering
 \includegraphics[width=0.15\textwidth]{./images/exp.jpg}\par\vspace{1cm}
  {\scshape\LARGE Universidad de Granada  \par}
  \vspace{1cm}
  {\scshape Proyecto Final\par}
  \vspace{1.5cm}
  {\huge\bfseries  Clasificación de radiografías torácicas para detección de COVID-19\par}
  \vspace{2cm}
  {\Large\itshape Alberto Luque Infante\\David Villar Martos\par}
  \vfill
  Quinto curso del Doble Grado de Ingeniería Informática y Matemáticas:\par
  Visión por Computador

  \vfill

% Bottom of the page
  {\large \today\par}
\end{titlepage}

\tableofcontents
\newpage
\section{Descripción del proyecto}

El proyecto consiste en abordar el problema de clasificación de radiografías torácicas con la intención de poder detectar casos positivos de COVID-19.\\

Partiendo de un modelo de red convolucional preentrenado con la base de datos ImageNet, lo utilizaremos como base para encontrar el mejor modelo que nos permita llevar a cabo esta tarea.\\

En la base de datos escogida se utilizan imágenes clasificadas en tres clases distintas: radiografías de pulmones sanos (NORMAL), radiografías de pulmones en casos positivos de COVID-19 (COVID) y radiografías de pulmones afectados con neumonía vírica (Viral Pneumonia). Tener este conjunto de datos es muy interesante puesto que no sólo tendremos que distinguir pulmones sanos de pulmones enfermos, sino también distinguir entre pulmones enfermos con neumonías víricas,  siendo distinto el tipo de virus que las provoca.\\

Por tanto, nuestro problema de clasificación va a consistir en implementar un modelo que nos permita clasificar una imagen como perteneciente a una de las 3 clases.\\

Primero haremos un análisis previo del problema para poder enfocarlo correctamente. A continuación comenzaremos con la lectura y preprocesamiento de los datos, conformando los conjuntos de entrenamiento y test.\\

Hemos elegido como modelo inicial preentrenado con ImageNet el modelo DenseNet121. Partiendo de una primera versión muy básica poco a poco le iremos añadiendo mejoras hasta encontrar el modelo más óptimo. Debido a la calidad de los resultados iniciales que hemos obtenido, como se verá posteriormente,  hemos creído conveniente orientar el estudio desarrollando los aspectos que comentaremos ahora.\\

Tras encontrar una versión razonablemente buena basándones en DenseNet121,  utilizaremos como base otros modelos conocidos preentrenados también en ImageNet, y emplearemos las misma modificaciones sobre ellos que las que hemos empleado sobre DenseNet121, para poder hacer valoraciones y comparaciones entre los mismos.\\

Finalmente, vamos realizar un análisis pormenorizado de los resultados obtenidos, visualizando los mapas de activación y mapas de calor con el objetivo de detectar qué zonas de las imágenes de entrada son más discriminativas de cara a realizar la clasificación. De esta forma podremos entender un poco mejor el funcionamiento del modelo que nos permitan valorarlo mejor y extraer conclusiones que puedan ayudar también al campo de la medicina.\\

Por último, comentaremos algunas propuestas o aspectos que se podrían mejorar en un futuro en base a la experiencia obtenida.\\

\newpage

\section{Análisis previo del problema}

El uso de la radiografía simple de tórax para la detección de patologías torácicas,  es una técnica muy efectiva y se considera la exploración base a realizar, debido a la gran cantidad de información que es capaz de aportar.  Se utilizan también otro tipo de técnicas,  tales como la tomografía computerizada,  resonancia magnética, radioscopia o ecografía, pero siempre como apoyo a la radiografía torácica.\\

Las proyecciones básicas de la radiografía simpla de tórax son la postoanterior (frontal) y la lateral.  Normalmente se practican en inspiración máxima y sostenida,  con el paciente en bipedestación.  La lectura de la radiografía de tórax por parte de un médico se hace siempre de forma sistemática, analizando secuencialmente y por orden: partes blandas, hueso, diafragma, mediastino, hilios pulmonares, pleura y parénquima pulmonar.\\

Clínicamente la neumonía se define como una consolidación pulmonar en una radiografía de tórax junto con signos y síntomas clínicos de infección respiratoria (fiebre, tos y expectoración). La consolidación pulmonar se ve reflejada a nivel radiológico en que aumenta la opacidad de los pulmones debido al cúmulo de otras sustancias más densas que el aire.\\

La consolidación pulmonar típica es debida con frecuencia a la presencia de una neumonía, pero este patrón radiológico no es específico de ella, ya que cualquier enfermedad que ocupe el espacio aéreo producirá la misma imagen. Por tanto, el examen radiológico es indispensable para el diagnóstico, pero carece de especificidad, por lo que el diagnóstico de la radiografía está ligada a la correlación clínica del paciente.\\

Las neumonías con respecto al nivel radiológico se dividen en 3 tipos: Neumonías lobulares y segmentarias, Bronconeumonías, y por último, Neumonías intersticiales. \\

La neumonía vírica COVID-19 entra dentro de este último grupo.  Desde el mes de diciembre del año 2019,  cuando se detectó el primer caso de COVID-19,  el virus se ha propagado a nivel mundial teniendo una alta tasa de contagio, considerándose una pandemia desde marzo de 2020.  Ha provocado desde entonces la mayor crisis sanitaria que se ha vivido globalmente en los tiempos modernos, y es una prioridad el intentar controlar la pandemia. Las radiografías torácicas permiten diagnosticar la enfermedad y entender mejor cómo afecta al organismo, por lo que consideramos muy relevante su estudio.  Sin embargo,  es posible confundir una neumonía vírica provocada por COVID-19 con otras neumonías víricas ocasionadas por otros patógenos. \\

Existen ya algunos estudios los cuales intentan encontrar patrones radiológicos que intenten diferenciar el COVID-19 de otros tipos de neumonías víricas, como en https://www.researchgate.net/publication/339839348\_Performance\_of\_radiologists\_in\_differentiating\_COVID-19\_from\_viral\_pneumonia\_on\_chest\_CT
 o en https://insightsimaging.springeropen.com/articles/10.1186/s13244-020-00933-z,  donde se encuentran dificultades para ello.\\

Las redes neuronales neuronales convolucionales constituyen uno de los métodos más potentes en la actualidad para clasificación de imágenes,  y tienen una eficacia probada en una enorme cantidad de contextos.  Estos métodos han sido también empleados en el contexto de la medicina, y particularmente, en el contexto de la radiología

(Yasaka K, Akai H, Kunimatsu A, Kiryu S, Abe O. Deep learning with convolutional neural network in radiology. Jpn J Radiol. 2018 Apr;36(4):257-272. doi: 10.1007/s11604-018-0726-3. Epub 2018 Mar 1. PMID: 29498017.)

Yamashita, R., Nishio, M., Do, R.K.G. et al. Convolutional neural networks: an overview and application in radiology. Insights Imaging 9, 611–629 (2018). https://doi.org/10.1007/s13244-018-0639-9

debido a su gran potencial de aprendizaje. Además, el hecho de usar redes neuronales preentrenadas puede ser muy útil no sólo a la hora de la rapidez a la hora de entrenarlas, sino debido a que se pueden aprovechar características aprendidas en otros contextos totalmente distintos para emplearlos en los problemas que nos conciernen.\\

Por tanto,  pretendemos con nuestro estudio desarrollar mediante redes neuronales preentrenadas, un método efectivo para clasificación de radiografías torácicas postoanteriores, que sea capaz de distinguir pacientes con pulmones sanos de pacientes con pulmones enfermos, por neumonías víricas, pero además, que sea capaz de distinguir entre neumonías víricas, centrándonos en la distinción de si el agente patógeno es el COVID-19 o no lo es, para un mejor diagnóstico, estudio y entendimiento de esta enfermedad.  Utilizando distintos tipos de redes conocidas, entrenadas con pesos en Imagenet, podremos hacer comparaciones entre ellas y seleccionar la que mejores resultados aporte. Además intentaremos visualizar las diferentes activaciones que tiene la red para entender un poco mejor las características que ha aprendido, y comprobar que focaliza realmente su atención en la zona pulmonar.  El entender qué partes de la radiografía considera la red más importantes a la hora de clasificar podría usarse para mejorar el diagnóstico de la enfermedad y dirigir posteriores estudios de cara a entender mejor cómo afecta la enfermedad al organismo y cómo se diferencia de otro tipo de neumonías víricas \\
 
\section{Base de Datos elegida para el problema}
\subsection{Información de la Base de Datos}
La base de datos elegida para el proyecto se encuentra en Kaggle :

https://www.kaggle.com/tawsifurrahman/covid19-radiography-database.\\

La base de datos, contiene un total de 3886 imágenes de radiografías torácicas postoanteriores en su actualización del 5 de enero de 2021 (versión 3). La distribución de imágenes por clases es de 1200 imágenes de la clase COVID, 1341 imagenes de la clase NORMAL y 1345 imágenes de la clase Viral Pneumonia.\\

El conjunto de imágenes del dataset proceden de distintas fuentes. Las imágenes de la clase COVID proceden de las siguientes fuentes:
\begin{itemize}
\item 400 radiografias torácicas frontales de la fuente de github : https://github.com/armiro/COVID-CXNet
\item 183  radiografias torácicas frontales del colegio médico de Hannover, Alemania: https://github.com/ml-workgroup/covid-19-image-repository/tree/master/png.
\item 617 radiografías torácicas frontales de entre las siguientes fuentes
\begin{itemize}
\item Sociedad Italiana de Radiología Médicahttps://sirm.org/category/senza-categoria/covid-19/

\item https://eurorad.org, base de datos operada por Sociedad Europea de Radiología (ESR)

\item https://github.com/ieee8023/covid-chestxray-dataset . Aquí las imágenes vienen de distintas fuentes públicas además de otras fuentes indirectas de hospitales y físicos.
\end{itemize}
\end{itemize}.\\

El resto de imágenes (imágenes de pulmones sanos e imágenes de pulmones con otro tipo de neumonías víricas) provienen del dataset https://github.com/ieee8023/covid-chestxray-dataset.  Las imágenes de este conjunto de datos provienen de un centro médico en Guangzhou.\\

No hemos encontrado evidencia,  ni de que se repitan imágenes en las imágenes con coronavirus,  cosa que podría haber sido posible por el hecho de que se han extraído de varias fuentes,  y de hecho hay varias fuentes que cogen imágenes de otras que aquí aparecen (la de 400 coge de sirm,  por ejemplo) ni de que haya varias imágenes correspondientes a un mismo paciente en distintos instantes de tiempo. \\

Este conjunto de imágenes podría considerarse un conjunto de datos pequeño en comparación con otros grandes conjuntos de datos ampliamente empleados en el campo de visión por computador como pudiera ser Imagenet, CIFAR,...  Sin embargo, debido a lo relativamente reciente que es la pandemia del COVID-19(1 año y 2 meses),  y debido a que son imágenes médicas, es complicado encontrar grandes conjuntos de datos para esta tarea. El conjunto de datos actual es el más extenso de entre todos los que hemos podido encontrar.

\subsection{Lectura de la Base de Datos}

Para que las clases estén equilibradas vamos a tomar el mismo número de imágenes por clase, 1200, luego tendremos un total de 3600 imágenes.  El que los datos estén perfectamente equilibrados en nuestro conjunto de datos no quiere decir que en la vida real se den en la misma proporción,  pero preferimos que sea así par no favorecer en la tarea de clasificación a ninguna clase frente a las demás.\\

Para leer las imágenes, creamos un vector de etiquetas con las 1200 etiquetas para cada clase y utilizamos la función implementada \textit{leerImagenes}.\\

Leemos las imágenes de cada clase haciendo interpolación bilineal para que tengan un tamaño de (224,224), que es el tamaño que tienen las imágenes de ImageNet.\\
\begin{lstlisting}
def leerImagenes(clases, num_imgs, path):

  imgs_covid = np.array([img_to_array(load_img(path + "/" + clases[i] + "/" + clases[i] + " (" + str(i+1) + ").png",
                                                target_size = (224, 224), interpolation="bilinear")) for i in range(0,num_imgs)])
  imgs_normal = np.array([img_to_array(load_img(path + "/" + clases[i] + "/" + clases[i] + " (" + str(i-num_imgs+1) + ").png",
                                                target_size = (224, 224), interpolation="bilinear")) for i in range(num_imgs,2*num_imgs)])
  imgs_viral = np.array([img_to_array(load_img(path + "/" + clases[i] + "/" + clases[i] + " (" + str(i-2*num_imgs+1) + ").png",
                                                target_size = (224, 224), interpolation="bilinear")) for i in range(2*num_imgs,3*num_imgs)])

  return imgs_covid, imgs_normal, imgs_viral
\end{lstlisting}

Un ejemplo de una imagen de cada clase es el siguiente:

\begin{figure}[H]
  \centering
  \begin{minipage}[b]{0.25\textwidth}
    \includegraphics[scale=0.65]{./images/ejemploCOVID}
	\caption{Imagen de la clase COVID}
  \end{minipage}
  \hfill
  \begin{minipage}[b]{0.25\textwidth}
    \includegraphics[scale=0.65]{./images/ejemploNEUMONIAVIRICA}
	\caption{Imagen de la clase Viral Pneumonia}
  \end{minipage}
    \hfill
    \begin{minipage}[b]{0.25\textwidth}
    \includegraphics[scale=0.65]{./images/ejemploNORMAL}
	\caption{Imagen de la clase NORMAL}
  \end{minipage}
\end{figure}


\subsection{Creación de los conjuntos de train y test}

Primero creamos un vector agrupando las imágenes de las tres clases y cambiamos las etiquetas por números enteros, asignando el valor 0 a la clase COVID, el 1 a la clase NORMAL y el 2 a la clase Viral Pneumonia.\\

A continuación, usando la función \textit{to\_categorical} obtenemos una representación binaria de las clases (la clase 0 pasa a ser [1,0,0], la 1 es [0,1,0] y la 2 es [0,0,1]).\\

Ahora, para dividir los datos entre entrenamiento y test vamos a optar por hacer una permutación aleatoria de las imágenes, y posteriormente dividiéndolas en proporción 80\%-20\% para entrenamiento y test respectivamente, obteniendo un total de 2880 imágenes para entrenar y 720 para test. Hemos decidido hacer esta partición de los datos debido en parte a que, como hemos hecho notar antes, no hemos encontrado ninguna evidencia de que haya imágenes dentro del conjunto de datos que correspondan al mismo paciente en diferentes instantes de tiempo que puedan hacer que la tarea de clasificación se vea influenciada por este hecho. De ser así, que haya imágenes para un mismo paciente que cayeran unas en entrenamiento y otras en test, harían que aumentara el sobreajuste y la varianza de la red, haciendo que probablemente no generalizara bien y diera peores resultados en test. También pensamos que hubiera sido bastante útil hacer una división de los datos de forma temporal, dejando una proporción de las imágenes tomadas más recientemente como las imágenes para test, puesto que la enfermedad del COVID-19 está en constante evolución y buscamos que la red sea útil para poder ayudar a clasificar casos actuales de COVID-19 (aunque entendemos que las imágenes de pulmones sanos y pulmones con otros tipos de neumonías víricas no varían tanto a lo largo del tiempo). Sin embargo no hemos podido hacer esta división ya que no poseemos información de cuando se han tomado las imágenes. Son por estos hecho por los que hemos optado hacer el tipo de partición que hemos hecho.

\section{Implementación de los modelos}

\subsection{DenseNet121}

DenseNet121 es el modelo base que hemos elegido para empezar, puesto que es un modelo potente y entendemos que podríamos dar unos resultados decentes desde un primer momento aplicando técnicas sobre este modelo. En las primeras versiones vamos a utilizar el modelo, con los pesos de ImageNet, solo como extractor de características, a partir del cuál añadiremos modificaciones. Finalmente, haremos un ajuste fino del modelo donde sí moveremos los pesos.

\subsubsection{Version 0: Extractor de características inicial, sin entrenamiento}

Primero vamos a utilizar el modelo DenseNet121 preentrenado con IMAGENET sin entrenar ningún peso. Partiendo de las características extraídas, simplemente vamos a añadir la capa softmax y comprobaremos que los resultados no son para nada buenos, pues todavía no hemos entrenado nuevos pesos, no estamos adaptando la red a nuestros datos en absoluto.

Extraemos las características:

\begin{lstlisting}
#Vamos a cargar el modelo antes del ultimo pooling, para usarlo como extractor de caracteristicas
feat_extractor = DenseNet121(include_top=False, weights='imagenet', pooling='avg')
feat_extractor.trainable = False

#Una vez tenemos el modelo, vamos a compilarlo usando un optimizador y una funcion de perdida
opt = SGD(lr=0.01, decay= 1e-6, momentum=0.9, nesterov=True)

feat_extractor.compile(optimizer=opt, loss="categorical_crossentropy", metrics=["acc"])

#Con esto tenemos el modelo de densenet hasta que nos deja los datos en forma de vector unidimensional
#de dimension 1024
feat_extractor = keras.Model(inputs=feat_extractor.inputs, outputs= feat_extractor.layers[-1].output)

feat_extractor.compile(optimizer=opt, loss="categorical_crossentropy", metrics=["acc"])

# Extraer las caracteristicas de las imagenes con el modelo anterior.
car_train = feat_extractor.predict(x_train, verbose=1)
car_test = feat_extractor.predict(x_test, verbose=1)
\end{lstlisting}

El optimizador que he hemos empleado es el mismo que hemos usado en las prácticas, que es el que viene en las diapositivas, un gradiente descendente estocástico, con un tasa de aprendizaje de 0.01, con un paámetro de decaimiento de dicha tasa de aprendizaje de 1e-6, un momento de 0.9, y el descenso de gradiente se hará mediante la metodología Nesterov, en la cual, primero se realiza un desplazamiento en la dirección del gradiente acumulativo que llevamos, y posteriormente se realiza el nuevo cálculo del gradiente desde la posición en la que estemos en el espacio de búsqueda.\\

Una vez tenemos las características, añadimos la capa softmax y hacemos las predicciones.

\begin{lstlisting}
inputs = keras.layers.Input(shape=[1024])
outputs = keras.layers.Dense(units=3, activation="softmax")(inputs)

dense_model = keras.Model(inputs=inputs, outputs=outputs)
opt = SGD(lr=0.01, decay= 1e-6, momentum=0.9, nesterov=True)
dense_model.compile(optimizer=opt, loss="categorical_crossentropy", metrics=["acc"])

#Clasificamos
y_preds = dense_model.predict(car_test, verbose=True)

print("La accuracy del modelo es: " + str(calcularAccuracy(y_test, y_preds)))

y_test_conf = np.argmax(y_test, axis=1)
y_preds = np.argmax(y_preds, axis=1)
print("La matriz de confusion de las predicciones ha sido: \n", confusion_matrix(y_test_conf, y_preds))
\end{lstlisting}


Los resultados los podemos observar en la siguiente tabla:

\begin{table}[H]
\centering
\begin{tabular}{|c|c|c|c|c|c|c|}
\hline
\rowcolor[rgb]{0.753,0.753,0.753}  \textbf{Modelo}  & \textbf{Loss}                               & \textbf{Accuracy}                           & \begin{tabular}[c]{@{}>{\cellcolor[rgb]{0.753,0.753,0.753}}c@{}}\textbf{Validation}\\\textbf{ Loss} \end{tabular} & \begin{tabular}[c]{@{}>{\cellcolor[rgb]{0.753,0.753,0.753}}c@{}}\textbf{Validation }\\\textbf{ Accuracy} \end{tabular} & \begin{tabular}[c]{@{}>{\cellcolor[rgb]{0.753,0.753,0.753}}c@{}}\textbf{Test }\\\textbf{ Accuracy} \end{tabular} & \begin{tabular}[c]{@{}>{\cellcolor[rgb]{0.753,0.753,0.753}}c@{}}\textbf{Número}\\\textbf{Épocas} \end{tabular}  \\
\hline
\rowcolor[rgb]{0.937,0.937,0.937} DN-V0             &                                             &                                             &                                                                                                                   &                                                                                                                        & \textcolor[rgb]{0.129,0.129,0.129}{0.3264}                                                                       &                                                                                                                 \\
\hline

\end{tabular}
\end{table}

La matriz de confusión de la clasificación ha sido la siguiente:

\begin{table}[htpb]
\begin{center}
\begin{tabular}{l|
>{\columncolor[HTML]{EFEFEF}}l |
>{\columncolor[HTML]{EFEFEF}}l |
>{\columncolor[HTML]{EFEFEF}}l |}
\hline
\multicolumn{1}{|l|}{\cellcolor[HTML]{C0C0C0}\textbf{V-COVID}}  & 235                                      & 0                                         & 0                                       \\ \hline
\multicolumn{1}{|l|}{\cellcolor[HTML]{C0C0C0}\textbf{V-Normal}} & 244                                      & 0                                         & 0                                       \\ \hline
\multicolumn{1}{|l|}{\cellcolor[HTML]{C0C0C0}\textbf{V-Neum.}}  & 241                                      & 0                                         & 0                                       \\ \hline
                                                                & \cellcolor[HTML]{C0C0C0}\textbf{P-COVID} & \cellcolor[HTML]{C0C0C0}\textbf{P-Normal} & \cellcolor[HTML]{C0C0C0}\textbf{P-Neum} \\ \cline{2-4} 
\end{tabular}
\end{center}
\end{table}



\subsubsection{Version 1: Extractor de características inicial, entrenando la capa de salida}

Ahora vamos a reentrenar sólo la capa de salida, sin ningún tratamiento adicional. La única diferencia respecto a la versión anterior es que sí que entrenamos la última capa:

\begin{lstlisting}
hist = dense_model.fit(car_train, y_train, batch_size=32, epochs=20, validation_split=0.1)
\end{lstlisting}

\begin{figure}[H]
  \centering
  \begin{minipage}[b]{0.45\textwidth}
    \includegraphics[scale=0.75]{./images/v1loss}
	\caption{Training-Validation Loss}
  \end{minipage}
  \hfill
  \begin{minipage}[b]{0.45\textwidth}
    \includegraphics[scale=0.75]{./images/v1acc}
	\caption{Training-Validation accuracy}
  \end{minipage}
\end{figure}

\begin{table}[H]
\centering
\begin{tabular}{|c|c|c|c|c|c|c|}
\hline
\rowcolor[rgb]{0.753,0.753,0.753}  \textbf{Modelo}  & \textbf{Loss}                               & \textbf{Accuracy}                           & \begin{tabular}[c]{@{}>{\cellcolor[rgb]{0.753,0.753,0.753}}c@{}}\textbf{Validation}\\\textbf{ Loss} \end{tabular} & \begin{tabular}[c]{@{}>{\cellcolor[rgb]{0.753,0.753,0.753}}c@{}}\textbf{Validation }\\\textbf{ Accuracy} \end{tabular} & \begin{tabular}[c]{@{}>{\cellcolor[rgb]{0.753,0.753,0.753}}c@{}}\textbf{Test }\\\textbf{ Accuracy} \end{tabular} & \begin{tabular}[c]{@{}>{\cellcolor[rgb]{0.753,0.753,0.753}}c@{}}\textbf{Número}\\\textbf{Épocas} \end{tabular}  \\
\hline
\rowcolor[rgb]{0.937,0.937,0.937} DN-V0             &                                             &                                             &                                                                                                                   &                                                                                                                        & \textcolor[rgb]{0.129,0.129,0.129}{0.3264}                                                                       &                                                                                                                 \\
\hline
\rowcolor{green} DN-V1                                               & \textcolor[rgb]{0.129,0.129,0.129}{0.6501}  & \textcolor[rgb]{0.129,0.129,0.129}{0.9588 } & \textcolor[rgb]{0.129,0.129,0.129}{3.1393}                                                                        & \textcolor[rgb]{0.129,0.129,0.129}{0.8333}                                                                             & \textcolor[rgb]{0.129,0.129,0.129}{0.8361}                                                                       & 20                                                                                                              \\
\hline

\end{tabular}
\end{table}

\begin{table}[htbp]
\begin{center}
\begin{tabular}{l|
>{\columncolor[HTML]{EFEFEF}}l |
>{\columncolor[HTML]{EFEFEF}}l |
>{\columncolor[HTML]{EFEFEF}}l |}
\hline
\multicolumn{1}{|l|}{\cellcolor[HTML]{C0C0C0}\textbf{V-COVID}}  & 231                                      & 1                                         & 3                                       \\ \hline
\multicolumn{1}{|l|}{\cellcolor[HTML]{C0C0C0}\textbf{V-Normal}} & 6                                        & 135                                       & 103                                     \\ \hline
\multicolumn{1}{|l|}{\cellcolor[HTML]{C0C0C0}\textbf{V-Neum.}}  & 5                                        & 0                                         & 236                                     \\ \hline
                                                                & \cellcolor[HTML]{C0C0C0}\textbf{P-COVID} & \cellcolor[HTML]{C0C0C0}\textbf{P-Normal} & \cellcolor[HTML]{C0C0C0}\textbf{P-Neum} \\ \cline{2-4} 
\end{tabular}
\end{center}
\end{table}

\subsubsection{Version 2: Data augmentation, más capas densas}

Seguimos empleando la red preentrenada como extractor de características pero se realiza un preprocesado más exhaustivo de imágenes, con image augmentation y normalización, además de hacer un modelo denso de mayor profundidad.\\

Además de hacer una normalización de las imágenes, para que tengan media cero y varianza 1, hemos detectado que hay imágenes que salen muy blancas y otras muy oscuras, luego debemos conseguir invarianza frente al brillo (añadiendo el parámetro brightness).\\

En cuanto al data augmentation, los pulmones en las radiografías salen siempre verticales, por lo que no tiene mucho sentido añadir rotaciones pero sí flips horizontales. Los zooms de ampliación también pueden ser interesantes, puesto que las imágenes tienen siempre los pulmones en la zona central, y podemos evitar posibles fuentes de ruido de letras que tienen algunas de las imágenes en las zonas laterales, que se perderán haciendo zoom.\\

Para aplicar todo esto, creamos un objeto de la clase ImageDataGenerator:

\begin{lstlisting}
#Image data generators

train_generator = ImageDataGenerator(featurewise_center = True,
                             featurewise_std_normalization = True,
                             validation_split=0.1,
                             horizontal_flip=True,
                             brightness_range=[0.8,1.25],
                             zoom_range=[1,1.2])
train_generator.fit(x_train)

test_generator = ImageDataGenerator(featurewise_center = True,
                             featurewise_std_normalization = True)

test_generator.fit(x_train)

it = train_generator.flow(x_train, batch_size=1)
\end{lstlisting}

Añadimos también alguna capa densa más para ganar profundidad:

\begin{lstlisting}
inputs = keras.layers.Input(shape=[1024])
layer1 = keras.layers.Dense(units=500, activation="relu")(inputs)
layer2 = keras.layers.Dense(units=200, activation="relu")(layer1)
outputs = keras.layers.Dense(units=3, activation="softmax")(layer2)
\end{lstlisting}


RESULTADOS..........

\begin{figure}[H]
  \centering
  \begin{minipage}[b]{0.45\textwidth}
    \includegraphics[scale=0.75]{./images/v2loss}
	\caption{Training-Validation Loss}
  \end{minipage}
  \hfill
  \begin{minipage}[b]{0.45\textwidth}
    \includegraphics[scale=0.75]{./images/v2acc}
	\caption{Training-Validation accuracy}
  \end{minipage}
\end{figure}

\begin{table}[H]
\centering
\begin{tabular}{|c|c|c|c|c|c|c|}
\hline
\rowcolor[rgb]{0.753,0.753,0.753}  \textbf{Modelo}  & \textbf{Loss}                               & \textbf{Accuracy}                           & \begin{tabular}[c]{@{}>{\cellcolor[rgb]{0.753,0.753,0.753}}c@{}}\textbf{Validation}\\\textbf{ Loss} \end{tabular} & \begin{tabular}[c]{@{}>{\cellcolor[rgb]{0.753,0.753,0.753}}c@{}}\textbf{Validation }\\\textbf{ Accuracy} \end{tabular} & \begin{tabular}[c]{@{}>{\cellcolor[rgb]{0.753,0.753,0.753}}c@{}}\textbf{Test }\\\textbf{ Accuracy} \end{tabular} & \begin{tabular}[c]{@{}>{\cellcolor[rgb]{0.753,0.753,0.753}}c@{}}\textbf{Número}\\\textbf{Épocas} \end{tabular}  \\
\hline
\rowcolor[rgb]{0.937,0.937,0.937} DN-V0             &                                             &                                             &                                                                                                                   &                                                                                                                        & \textcolor[rgb]{0.129,0.129,0.129}{0.3264}                                                                       &                                                                                                                 \\
\hline
DN-V1                                               & \textcolor[rgb]{0.129,0.129,0.129}{0.6501}  & \textcolor[rgb]{0.129,0.129,0.129}{0.9588 } & \textcolor[rgb]{0.129,0.129,0.129}{3.1393}                                                                        & \textcolor[rgb]{0.129,0.129,0.129}{0.8333}                                                                             & \textcolor[rgb]{0.129,0.129,0.129}{0.8361}                                                                       & 20                                                                                                              \\
\hline
\rowcolor{green} DN-V2                                               & \textcolor[rgb]{0.129,0.129,0.129}{0.0109 } & \textcolor[rgb]{0.129,0.129,0.129}{0.9973 } & \textcolor[rgb]{0.129,0.129,0.129}{0.1146 }                                                                       & \textcolor[rgb]{0.129,0.129,0.129}{0.9583}                                                                             & \textcolor[rgb]{0.129,0.129,0.129}{0.9597}                                                                       & 20                                                                                                              \\
\hline


\end{tabular}
\end{table}


\begin{table}[htbp]
\begin{center}
\begin{tabular}{l|
>{\columncolor[HTML]{EFEFEF}}l |
>{\columncolor[HTML]{EFEFEF}}l |
>{\columncolor[HTML]{EFEFEF}}l |}
\hline
\multicolumn{1}{|l|}{\cellcolor[HTML]{C0C0C0}\textbf{V-COVID}}  & 230                                      & 3                                         & 2                                       \\ \hline
\multicolumn{1}{|l|}{\cellcolor[HTML]{C0C0C0}\textbf{V-Normal}} & 1                                        & 238                                       & 5                                       \\ \hline
\multicolumn{1}{|l|}{\cellcolor[HTML]{C0C0C0}\textbf{V-Neum.}}  & 1                                        & 17                                        & 223                                     \\ \hline
                                                                & \cellcolor[HTML]{C0C0C0}\textbf{P-COVID} & \cellcolor[HTML]{C0C0C0}\textbf{P-Normal} & \cellcolor[HTML]{C0C0C0}\textbf{P-Neum} \\ \cline{2-4} 
\end{tabular}
\end{center}
\end{table}



En esta última versión puede ser que tengamos un poco de overfitting puesto que el validation loss no decrece al mismo ritmo que el training loss. Regularizamos añadiendo una capa de Dropout.

\begin{lstlisting}
inputs = keras.layers.Input(shape=[1024])
layer1 = keras.layers.Dense(units=500, activation="relu")(inputs)
layer2 = keras.layers.Dropout(0.5)(layer1)
layer3 = keras.layers.Dense(units=200, activation="relu")(layer2)
outputs = keras.layers.Dense(units=3, activation="softmax")(layer3)
\end{lstlisting}

RESULTADOS................

\begin{figure}[H]
  \centering
  \begin{minipage}[b]{0.45\textwidth}
    \includegraphics[scale=0.75]{./images/v2-2loss}
	\caption{Training-Validation Loss}
  \end{minipage}
  \hfill
  \begin{minipage}[b]{0.45\textwidth}
    \includegraphics[scale=0.75]{./images/v2-2acc}
	\caption{Training-Validation accuracy}
  \end{minipage}
\end{figure}

\begin{table}[H]
\centering
\begin{tabular}{|c|c|c|c|c|c|c|}
\hline
\rowcolor[rgb]{0.753,0.753,0.753}  \textbf{Modelo}  & \textbf{Loss}                               & \textbf{Accuracy}                           & \begin{tabular}[c]{@{}>{\cellcolor[rgb]{0.753,0.753,0.753}}c@{}}\textbf{Validation}\\\textbf{ Loss} \end{tabular} & \begin{tabular}[c]{@{}>{\cellcolor[rgb]{0.753,0.753,0.753}}c@{}}\textbf{Validation }\\\textbf{ Accuracy} \end{tabular} & \begin{tabular}[c]{@{}>{\cellcolor[rgb]{0.753,0.753,0.753}}c@{}}\textbf{Test }\\\textbf{ Accuracy} \end{tabular} & \begin{tabular}[c]{@{}>{\cellcolor[rgb]{0.753,0.753,0.753}}c@{}}\textbf{Número}\\\textbf{Épocas} \end{tabular}  \\
\hline
\rowcolor[rgb]{0.937,0.937,0.937} DN-V0             &                                             &                                             &                                                                                                                   &                                                                                                                        & \textcolor[rgb]{0.129,0.129,0.129}{0.3264}                                                                       &                                                                                                                 \\
\hline
DN-V1                                               & \textcolor[rgb]{0.129,0.129,0.129}{0.6501}  & \textcolor[rgb]{0.129,0.129,0.129}{0.9588 } & \textcolor[rgb]{0.129,0.129,0.129}{3.1393}                                                                        & \textcolor[rgb]{0.129,0.129,0.129}{0.8333}                                                                             & \textcolor[rgb]{0.129,0.129,0.129}{0.8361}                                                                       & 20                                                                                                              \\
\hline
DN-V2                                               & \textcolor[rgb]{0.129,0.129,0.129}{0.0109 } & \textcolor[rgb]{0.129,0.129,0.129}{0.9973 } & \textcolor[rgb]{0.129,0.129,0.129}{0.1146 }                                                                       & \textcolor[rgb]{0.129,0.129,0.129}{0.9583}                                                                             & \textcolor[rgb]{0.129,0.129,0.129}{0.9597}                                                                       & 20                                                                                                              \\
\hline
\rowcolor{green} DN-V2-reg                                           & \textcolor[rgb]{0.129,0.129,0.129}{0.0883 } & \textcolor[rgb]{0.129,0.129,0.129}{0.9683 } & \textcolor[rgb]{0.129,0.129,0.129}{0.1494 }                                                                       & \textcolor[rgb]{0.129,0.129,0.129}{0.9583}                                                                             & \textcolor[rgb]{0.129,0.129,0.129}{0.9569}                                                                       & 20                                                                                                              \\
\hline

\end{tabular}
\end{table}

\begin{table}[htbp]
\begin{center}
\begin{tabular}{l|
>{\columncolor[HTML]{EFEFEF}}l |
>{\columncolor[HTML]{EFEFEF}}l |
>{\columncolor[HTML]{EFEFEF}}l |}
\hline
\multicolumn{1}{|l|}{\cellcolor[HTML]{C0C0C0}\textbf{V-COVID}}  & 226                                      & 2                                         & 7                                       \\ \hline
\multicolumn{1}{|l|}{\cellcolor[HTML]{C0C0C0}\textbf{V-Normal}} & 0                                        & 233                                       & 11                                      \\ \hline
\multicolumn{1}{|l|}{\cellcolor[HTML]{C0C0C0}\textbf{V-Neum.}}  & 0                                        & 11                                        & 230                                     \\ \hline
                                                                & \cellcolor[HTML]{C0C0C0}\textbf{P-COVID} & \cellcolor[HTML]{C0C0C0}\textbf{P-Normal} & \cellcolor[HTML]{C0C0C0}\textbf{P-Neum} \\ \cline{2-4} 
\end{tabular}
\end{center}
\end{table}

\subsubsection{Version 3: Fine tunning}

Vamos a probar ahora a hacer un ajuste fino de la red completa, tomando como pesos iniciales los de imagenet. Ahora ya no solo vamos a utilizar DenseNet como extractor de características, sino que, en vez de estar la red congelada, vamos a ir moviendo sus pesos con el entrenamiento.

\begin{figure}[H]
  \centering
  \begin{minipage}[b]{0.45\textwidth}
    \includegraphics[scale=0.75]{./images/v3loss}
	\caption{Training-Validation Loss}
  \end{minipage}
  \hfill
  \begin{minipage}[b]{0.45\textwidth}
    \includegraphics[scale=0.75]{./images/v3acc}
	\caption{Training-Validation accuracy}
  \end{minipage}
\end{figure}

\begin{table}[H]
\centering
\begin{tabular}{|c|c|c|c|c|c|c|}
\hline
\rowcolor[rgb]{0.753,0.753,0.753}  \textbf{Modelo}  & \textbf{Loss}                               & \textbf{Accuracy}                           & \begin{tabular}[c]{@{}>{\cellcolor[rgb]{0.753,0.753,0.753}}c@{}}\textbf{Validation}\\\textbf{ Loss} \end{tabular} & \begin{tabular}[c]{@{}>{\cellcolor[rgb]{0.753,0.753,0.753}}c@{}}\textbf{Validation }\\\textbf{ Accuracy} \end{tabular} & \begin{tabular}[c]{@{}>{\cellcolor[rgb]{0.753,0.753,0.753}}c@{}}\textbf{Test }\\\textbf{ Accuracy} \end{tabular} & \begin{tabular}[c]{@{}>{\cellcolor[rgb]{0.753,0.753,0.753}}c@{}}\textbf{Número}\\\textbf{Épocas} \end{tabular}  \\
\hline
\rowcolor[rgb]{0.937,0.937,0.937} DN-V0             &                                             &                                             &                                                                                                                   &                                                                                                                        & \textcolor[rgb]{0.129,0.129,0.129}{0.3264}                                                                       &                                                                                                                 \\
\hline
DN-V1                                               & \textcolor[rgb]{0.129,0.129,0.129}{0.6501}  & \textcolor[rgb]{0.129,0.129,0.129}{0.9588 } & \textcolor[rgb]{0.129,0.129,0.129}{3.1393}                                                                        & \textcolor[rgb]{0.129,0.129,0.129}{0.8333}                                                                             & \textcolor[rgb]{0.129,0.129,0.129}{0.8361}                                                                       & 20                                                                                                              \\
\hline
DN-V2                                               & \textcolor[rgb]{0.129,0.129,0.129}{0.0109 } & \textcolor[rgb]{0.129,0.129,0.129}{0.9973 } & \textcolor[rgb]{0.129,0.129,0.129}{0.1146 }                                                                       & \textcolor[rgb]{0.129,0.129,0.129}{0.9583}                                                                             & \textcolor[rgb]{0.129,0.129,0.129}{0.9597}                                                                       & 20                                                                                                              \\
\hline
DN-V2-reg                                           & \textcolor[rgb]{0.129,0.129,0.129}{0.0883 } & \textcolor[rgb]{0.129,0.129,0.129}{0.9683 } & \textcolor[rgb]{0.129,0.129,0.129}{0.1494 }                                                                       & \textcolor[rgb]{0.129,0.129,0.129}{0.9583}                                                                             & \textcolor[rgb]{0.129,0.129,0.129}{0.9569}                                                                       & 20                                                                                                              \\
\hline
\rowcolor{green} DN-V3                                               & \textcolor[rgb]{0.129,0.129,0.129}{0.0089 } & \textcolor[rgb]{0.129,0.129,0.129}{0.9964 } & \textcolor[rgb]{0.129,0.129,0.129}{0.0924 }                                                                       & \textcolor[rgb]{0.129,0.129,0.129}{0.9826}                                                                             & \textcolor[rgb]{0.129,0.129,0.129}{0.9736}                                                                       & 20                                                                                                              \\
\hline


\end{tabular}
\end{table}

\begin{table}[htbp]
\begin{center}
\begin{tabular}{l|
>{\columncolor[HTML]{EFEFEF}}l |
>{\columncolor[HTML]{EFEFEF}}l |
>{\columncolor[HTML]{EFEFEF}}l |}
\hline
\multicolumn{1}{|l|}{\cellcolor[HTML]{C0C0C0}\textbf{V-COVID}}  & 235                                      & 0                                         & 0                                       \\ \hline
\multicolumn{1}{|l|}{\cellcolor[HTML]{C0C0C0}\textbf{V-Normal}} & 0                                        & 230                                       & 14                                      \\ \hline
\multicolumn{1}{|l|}{\cellcolor[HTML]{C0C0C0}\textbf{V-Neum.}}  & 4                                        & 1                                         & 236                                     \\ \hline
                                                                & \cellcolor[HTML]{C0C0C0}\textbf{P-COVID} & \cellcolor[HTML]{C0C0C0}\textbf{P-Normal} & \cellcolor[HTML]{C0C0C0}\textbf{P-Neum} \\ \cline{2-4} 
\end{tabular}
\end{center}
\end{table}



\subsection{ResNet50}


\begin{figure}[H]
  \centering
  \begin{minipage}[b]{0.45\textwidth}
    \includegraphics[scale=0.75]{./images/resnet1loss}
	\caption{Training-Validation Loss}
  \end{minipage}
  \hfill
  \begin{minipage}[b]{0.45\textwidth}
    \includegraphics[scale=0.75]{./images/resnet1acc}
	\caption{Training-Validation accuracy}
  \end{minipage}
\end{figure}



\begin{table}[H]
\centering
\begin{tabular}{|c|c|c|c|c|c|c|}
\hline
\rowcolor[rgb]{0.753,0.753,0.753}  \textbf{Modelo}  & \textbf{Loss}                               & \textbf{Accuracy}                           & \begin{tabular}[c]{@{}>{\cellcolor[rgb]{0.753,0.753,0.753}}c@{}}\textbf{Validation}\\\textbf{ Loss} \end{tabular} & \begin{tabular}[c]{@{}>{\cellcolor[rgb]{0.753,0.753,0.753}}c@{}}\textbf{Validation }\\\textbf{ Accuracy} \end{tabular} & \begin{tabular}[c]{@{}>{\cellcolor[rgb]{0.753,0.753,0.753}}c@{}}\textbf{Test }\\\textbf{ Accuracy} \end{tabular} & \begin{tabular}[c]{@{}>{\cellcolor[rgb]{0.753,0.753,0.753}}c@{}}\textbf{Número}\\\textbf{Épocas} \end{tabular}  \\
\hline
\rowcolor[rgb]{0.937,0.937,0.937} DN-V0             &                                             &                                             &                                                                                                                   &                                                                                                                        & \textcolor[rgb]{0.129,0.129,0.129}{0.3264}                                                                       &                                                                                                                 \\
\hline
DN-V1                                               & \textcolor[rgb]{0.129,0.129,0.129}{0.6501}  & \textcolor[rgb]{0.129,0.129,0.129}{0.9588 } & \textcolor[rgb]{0.129,0.129,0.129}{3.1393}                                                                        & \textcolor[rgb]{0.129,0.129,0.129}{0.8333}                                                                             & \textcolor[rgb]{0.129,0.129,0.129}{0.8361}                                                                       & 20                                                                                                              \\
\hline
DN-V2                                               & \textcolor[rgb]{0.129,0.129,0.129}{0.0109 } & \textcolor[rgb]{0.129,0.129,0.129}{0.9973 } & \textcolor[rgb]{0.129,0.129,0.129}{0.1146 }                                                                       & \textcolor[rgb]{0.129,0.129,0.129}{0.9583}                                                                             & \textcolor[rgb]{0.129,0.129,0.129}{0.9597}                                                                       & 20                                                                                                              \\
\hline
DN-V2-reg                                           & \textcolor[rgb]{0.129,0.129,0.129}{0.0883 } & \textcolor[rgb]{0.129,0.129,0.129}{0.9683 } & \textcolor[rgb]{0.129,0.129,0.129}{0.1494 }                                                                       & \textcolor[rgb]{0.129,0.129,0.129}{0.9583}                                                                             & \textcolor[rgb]{0.129,0.129,0.129}{0.9569}                                                                       & 20                                                                                                              \\
\hline
DN-V3                                               & \textcolor[rgb]{0.129,0.129,0.129}{0.0089 } & \textcolor[rgb]{0.129,0.129,0.129}{0.9964 } & \textcolor[rgb]{0.129,0.129,0.129}{0.0924 }                                                                       & \textcolor[rgb]{0.129,0.129,0.129}{0.9826}                                                                             & \textcolor[rgb]{0.129,0.129,0.129}{0.9736}                                                                       & 20                                                                                                              \\
\hline
\rowcolor{green} ResNet                                              & \textcolor[rgb]{0.129,0.129,0.129}{0.2519 } & \textcolor[rgb]{0.129,0.129,0.129}{0.9063 } & \textcolor[rgb]{0.129,0.129,0.129}{0.2134 }                                                                       & \textcolor[rgb]{0.129,0.129,0.129}{0.9236}                                                                             & \textcolor[rgb]{0.129,0.129,0.129}{0.8805}                                                                       & 20                                                                                                              \\
\hline


\end{tabular}
\end{table}


\begin{table}[htbp]
\begin{center}
\begin{tabular}{l|
>{\columncolor[HTML]{EFEFEF}}l |
>{\columncolor[HTML]{EFEFEF}}l |
>{\columncolor[HTML]{EFEFEF}}l |}
\hline
\multicolumn{1}{|l|}{\cellcolor[HTML]{C0C0C0}\textbf{V-COVID}}  & 234                                      & 1                                         & 0                                       \\ \hline
\multicolumn{1}{|l|}{\cellcolor[HTML]{C0C0C0}\textbf{V-Normal}} & 15                                       & 219                                       & 10                                      \\ \hline
\multicolumn{1}{|l|}{\cellcolor[HTML]{C0C0C0}\textbf{V-Neum.}}  & 46                                       & 14                                        & 181                                     \\ \hline
                                                                & \cellcolor[HTML]{C0C0C0}\textbf{P-COVID} & \cellcolor[HTML]{C0C0C0}\textbf{P-Normal} & \cellcolor[HTML]{C0C0C0}\textbf{P-Neum} \\ \cline{2-4} 
\end{tabular}
\end{center}
\end{table}

\begin{figure}[H]
  \centering
  \begin{minipage}[b]{0.45\textwidth}
    \includegraphics[scale=0.75]{./images/resnet2loss}
	\caption{Training-Validation Loss}
  \end{minipage}
  \hfill
  \begin{minipage}[b]{0.45\textwidth}
    \includegraphics[scale=0.75]{./images/resnet2acc}
	\caption{Training-Validation accuracy}
  \end{minipage}
\end{figure}

\begin{table}[H]
\centering
\begin{tabular}{|c|c|c|c|c|c|c|}
\hline
\rowcolor[rgb]{0.753,0.753,0.753}  \textbf{Modelo}  & \textbf{Loss}                               & \textbf{Accuracy}                           & \begin{tabular}[c]{@{}>{\cellcolor[rgb]{0.753,0.753,0.753}}c@{}}\textbf{Validation}\\\textbf{ Loss} \end{tabular} & \begin{tabular}[c]{@{}>{\cellcolor[rgb]{0.753,0.753,0.753}}c@{}}\textbf{Validation }\\\textbf{ Accuracy} \end{tabular} & \begin{tabular}[c]{@{}>{\cellcolor[rgb]{0.753,0.753,0.753}}c@{}}\textbf{Test }\\\textbf{ Accuracy} \end{tabular} & \begin{tabular}[c]{@{}>{\cellcolor[rgb]{0.753,0.753,0.753}}c@{}}\textbf{Número}\\\textbf{Épocas} \end{tabular}  \\
\hline
\rowcolor[rgb]{0.937,0.937,0.937} DN-V0             &                                             &                                             &                                                                                                                   &                                                                                                                        & \textcolor[rgb]{0.129,0.129,0.129}{0.3264}                                                                       &                                                                                                                 \\
\hline
DN-V1                                               & \textcolor[rgb]{0.129,0.129,0.129}{0.6501}  & \textcolor[rgb]{0.129,0.129,0.129}{0.9588 } & \textcolor[rgb]{0.129,0.129,0.129}{3.1393}                                                                        & \textcolor[rgb]{0.129,0.129,0.129}{0.8333}                                                                             & \textcolor[rgb]{0.129,0.129,0.129}{0.8361}                                                                       & 20                                                                                                              \\
\hline
DN-V2                                               & \textcolor[rgb]{0.129,0.129,0.129}{0.0109 } & \textcolor[rgb]{0.129,0.129,0.129}{0.9973 } & \textcolor[rgb]{0.129,0.129,0.129}{0.1146 }                                                                       & \textcolor[rgb]{0.129,0.129,0.129}{0.9583}                                                                             & \textcolor[rgb]{0.129,0.129,0.129}{0.9597}                                                                       & 20                                                                                                              \\
\hline
DN-V2-reg                                           & \textcolor[rgb]{0.129,0.129,0.129}{0.0883 } & \textcolor[rgb]{0.129,0.129,0.129}{0.9683 } & \textcolor[rgb]{0.129,0.129,0.129}{0.1494 }                                                                       & \textcolor[rgb]{0.129,0.129,0.129}{0.9583}                                                                             & \textcolor[rgb]{0.129,0.129,0.129}{0.9569}                                                                       & 20                                                                                                              \\
\hline
DN-V3                                               & \textcolor[rgb]{0.129,0.129,0.129}{0.0089 } & \textcolor[rgb]{0.129,0.129,0.129}{0.9964 } & \textcolor[rgb]{0.129,0.129,0.129}{0.0924 }                                                                       & \textcolor[rgb]{0.129,0.129,0.129}{0.9826}                                                                             & \textcolor[rgb]{0.129,0.129,0.129}{0.9736}                                                                       & 20                                                                                                              \\
\hline
ResNet                                              & \textcolor[rgb]{0.129,0.129,0.129}{0.2519 } & \textcolor[rgb]{0.129,0.129,0.129}{0.9063 } & \textcolor[rgb]{0.129,0.129,0.129}{0.2134 }                                                                       & \textcolor[rgb]{0.129,0.129,0.129}{0.9236}                                                                             & \textcolor[rgb]{0.129,0.129,0.129}{0.8805}                                                                       & 20                                                                                                              \\
\hline
\rowcolor{green} ResNet-FT                                           & \textcolor[rgb]{0.129,0.129,0.129}{0.0102 } & \textcolor[rgb]{0.129,0.129,0.129}{0.9952 } & \textcolor[rgb]{0.129,0.129,0.129}{0.5594 }                                                                       & \textcolor[rgb]{0.129,0.129,0.129}{0.8576}                                                                             & \textcolor[rgb]{0.129,0.129,0.129}{0.8736}                                                                       & 20                                                                                                              \\
\hline


\end{tabular}
\end{table}


\begin{table}[htbp]
\begin{center}
\begin{tabular}{l|
>{\columncolor[HTML]{EFEFEF}}l |
>{\columncolor[HTML]{EFEFEF}}l |
>{\columncolor[HTML]{EFEFEF}}l |}
\hline
\multicolumn{1}{|l|}{\cellcolor[HTML]{C0C0C0}\textbf{V-COVID}}  & 173                                      & 0                                         & 62                                      \\ \hline
\multicolumn{1}{|l|}{\cellcolor[HTML]{C0C0C0}\textbf{V-Normal}} & 2                                        & 220                                       & 22                                      \\ \hline
\multicolumn{1}{|l|}{\cellcolor[HTML]{C0C0C0}\textbf{V-Neum.}}  & 0                                        & 5                                         & 236                                     \\ \hline
                                                                & \cellcolor[HTML]{C0C0C0}\textbf{P-COVID} & \cellcolor[HTML]{C0C0C0}\textbf{P-Normal} & \cellcolor[HTML]{C0C0C0}\textbf{P-Neum} \\ \cline{2-4} 
\end{tabular}
\end{center}
\end{table}


\subsection{VGG16}

\begin{figure}[H]
  \centering
  \begin{minipage}[b]{0.45\textwidth}
    \includegraphics[scale=0.75]{./images/vggloss}
	\caption{Training-Validation Loss}
  \end{minipage}
  \hfill
  \begin{minipage}[b]{0.45\textwidth}
    \includegraphics[scale=0.75]{./images/vggacc}
	\caption{Training-Validation accuracy}
  \end{minipage}
\end{figure}

\begin{table}[H]
\centering
\begin{tabular}{|c|c|c|c|c|c|c|}
\hline
\rowcolor[rgb]{0.753,0.753,0.753}  \textbf{Modelo}  & \textbf{Loss}                               & \textbf{Accuracy}                           & \begin{tabular}[c]{@{}>{\cellcolor[rgb]{0.753,0.753,0.753}}c@{}}\textbf{Validation}\\\textbf{ Loss} \end{tabular} & \begin{tabular}[c]{@{}>{\cellcolor[rgb]{0.753,0.753,0.753}}c@{}}\textbf{Validation }\\\textbf{ Accuracy} \end{tabular} & \begin{tabular}[c]{@{}>{\cellcolor[rgb]{0.753,0.753,0.753}}c@{}}\textbf{Test }\\\textbf{ Accuracy} \end{tabular} & \begin{tabular}[c]{@{}>{\cellcolor[rgb]{0.753,0.753,0.753}}c@{}}\textbf{Número}\\\textbf{Épocas} \end{tabular}  \\
\hline
\rowcolor[rgb]{0.937,0.937,0.937} DN-V0             &                                             &                                             &                                                                                                                   &                                                                                                                        & \textcolor[rgb]{0.129,0.129,0.129}{0.3264}                                                                       &                                                                                                                 \\
\hline
DN-V1                                               & \textcolor[rgb]{0.129,0.129,0.129}{0.6501}  & \textcolor[rgb]{0.129,0.129,0.129}{0.9588 } & \textcolor[rgb]{0.129,0.129,0.129}{3.1393}                                                                        & \textcolor[rgb]{0.129,0.129,0.129}{0.8333}                                                                             & \textcolor[rgb]{0.129,0.129,0.129}{0.8361}                                                                       & 20                                                                                                              \\
\hline
DN-V2                                               & \textcolor[rgb]{0.129,0.129,0.129}{0.0109 } & \textcolor[rgb]{0.129,0.129,0.129}{0.9973 } & \textcolor[rgb]{0.129,0.129,0.129}{0.1146 }                                                                       & \textcolor[rgb]{0.129,0.129,0.129}{0.9583}                                                                             & \textcolor[rgb]{0.129,0.129,0.129}{0.9597}                                                                       & 20                                                                                                              \\
\hline
DN-V2-reg                                           & \textcolor[rgb]{0.129,0.129,0.129}{0.0883 } & \textcolor[rgb]{0.129,0.129,0.129}{0.9683 } & \textcolor[rgb]{0.129,0.129,0.129}{0.1494 }                                                                       & \textcolor[rgb]{0.129,0.129,0.129}{0.9583}                                                                             & \textcolor[rgb]{0.129,0.129,0.129}{0.9569}                                                                       & 20                                                                                                              \\
\hline
DN-V3                                               & \textcolor[rgb]{0.129,0.129,0.129}{0.0089 } & \textcolor[rgb]{0.129,0.129,0.129}{0.9964 } & \textcolor[rgb]{0.129,0.129,0.129}{0.0924 }                                                                       & \textcolor[rgb]{0.129,0.129,0.129}{0.9826}                                                                             & \textcolor[rgb]{0.129,0.129,0.129}{0.9736}                                                                       & 20                                                                                                              \\
\hline
ResNet                                              & \textcolor[rgb]{0.129,0.129,0.129}{0.2519 } & \textcolor[rgb]{0.129,0.129,0.129}{0.9063 } & \textcolor[rgb]{0.129,0.129,0.129}{0.2134 }                                                                       & \textcolor[rgb]{0.129,0.129,0.129}{0.9236}                                                                             & \textcolor[rgb]{0.129,0.129,0.129}{0.8805}                                                                       & 20                                                                                                              \\
\hline
ResNet-FT                                           & \textcolor[rgb]{0.129,0.129,0.129}{0.0102 } & \textcolor[rgb]{0.129,0.129,0.129}{0.9952 } & \textcolor[rgb]{0.129,0.129,0.129}{0.5594 }                                                                       & \textcolor[rgb]{0.129,0.129,0.129}{0.8576}                                                                             & \textcolor[rgb]{0.129,0.129,0.129}{0.8736}                                                                       & 20                                                                                                              \\
\hline
\rowcolor{green} VGG16                                               & \textcolor[rgb]{0.129,0.129,0.129}{0.1057 } & \textcolor[rgb]{0.129,0.129,0.129}{0.9626 } & \textcolor[rgb]{0.129,0.129,0.129}{0.1063 }                                                                       & \textcolor[rgb]{0.129,0.129,0.129}{0.9618}                                                                             & \textcolor[rgb]{0.129,0.129,0.129}{0.9375}                                                                       & 20                                                                                                              \\
\hline

\end{tabular}
\end{table}

\begin{table}[htbp]
\begin{center}
\begin{tabular}{l|
>{\columncolor[HTML]{EFEFEF}}l |
>{\columncolor[HTML]{EFEFEF}}l |
>{\columncolor[HTML]{EFEFEF}}l |}
\hline
\multicolumn{1}{|l|}{\cellcolor[HTML]{C0C0C0}\textbf{V-COVID}}  & 234                                      & 0                                         & 1                                       \\ \hline
\multicolumn{1}{|l|}{\cellcolor[HTML]{C0C0C0}\textbf{V-Normal}} & 2                                        & 232                                       & 10                                      \\ \hline
\multicolumn{1}{|l|}{\cellcolor[HTML]{C0C0C0}\textbf{V-Neum.}}  & 8                                        & 24                                        & 209                                     \\ \hline
                                                                & \cellcolor[HTML]{C0C0C0}\textbf{P-COVID} & \cellcolor[HTML]{C0C0C0}\textbf{P-Normal} & \cellcolor[HTML]{C0C0C0}\textbf{P-Neum} \\ \cline{2-4} 
\end{tabular}
\end{center}
\end{table}


\begin{figure}[H]
  \centering
  \begin{minipage}[b]{0.45\textwidth}
    \includegraphics[scale=0.75]{./images/vgg2loss}
	\caption{Training-Validation Loss}
  \end{minipage}
  \hfill
  \begin{minipage}[b]{0.45\textwidth}
    \includegraphics[scale=0.75]{./images/vgg2acc}
	\caption{Training-Validation accuracy}
  \end{minipage}
\end{figure}

\begin{table}[H]
\centering
\begin{tabular}{|c|c|c|c|c|c|c|}
\hline
\rowcolor[rgb]{0.753,0.753,0.753}  \textbf{Modelo}  & \textbf{Loss}                               & \textbf{Accuracy}                           & \begin{tabular}[c]{@{}>{\cellcolor[rgb]{0.753,0.753,0.753}}c@{}}\textbf{Validation}\\\textbf{ Loss} \end{tabular} & \begin{tabular}[c]{@{}>{\cellcolor[rgb]{0.753,0.753,0.753}}c@{}}\textbf{Validation }\\\textbf{ Accuracy} \end{tabular} & \begin{tabular}[c]{@{}>{\cellcolor[rgb]{0.753,0.753,0.753}}c@{}}\textbf{Test }\\\textbf{ Accuracy} \end{tabular} & \begin{tabular}[c]{@{}>{\cellcolor[rgb]{0.753,0.753,0.753}}c@{}}\textbf{Número}\\\textbf{Épocas} \end{tabular}  \\
\hline
\rowcolor[rgb]{0.937,0.937,0.937} DN-V0             &                                             &                                             &                                                                                                                   &                                                                                                                        & \textcolor[rgb]{0.129,0.129,0.129}{0.3264}                                                                       &                                                                                                                 \\
\hline
DN-V1                                               & \textcolor[rgb]{0.129,0.129,0.129}{0.6501}  & \textcolor[rgb]{0.129,0.129,0.129}{0.9588 } & \textcolor[rgb]{0.129,0.129,0.129}{3.1393}                                                                        & \textcolor[rgb]{0.129,0.129,0.129}{0.8333}                                                                             & \textcolor[rgb]{0.129,0.129,0.129}{0.8361}                                                                       & 20                                                                                                              \\
\hline
DN-V2                                               & \textcolor[rgb]{0.129,0.129,0.129}{0.0109 } & \textcolor[rgb]{0.129,0.129,0.129}{0.9973 } & \textcolor[rgb]{0.129,0.129,0.129}{0.1146 }                                                                       & \textcolor[rgb]{0.129,0.129,0.129}{0.9583}                                                                             & \textcolor[rgb]{0.129,0.129,0.129}{0.9597}                                                                       & 20                                                                                                              \\
\hline
DN-V2-reg                                           & \textcolor[rgb]{0.129,0.129,0.129}{0.0883 } & \textcolor[rgb]{0.129,0.129,0.129}{0.9683 } & \textcolor[rgb]{0.129,0.129,0.129}{0.1494 }                                                                       & \textcolor[rgb]{0.129,0.129,0.129}{0.9583}                                                                             & \textcolor[rgb]{0.129,0.129,0.129}{0.9569}                                                                       & 20                                                                                                              \\
\hline
DN-V3                                               & \textcolor[rgb]{0.129,0.129,0.129}{0.0089 } & \textcolor[rgb]{0.129,0.129,0.129}{0.9964 } & \textcolor[rgb]{0.129,0.129,0.129}{0.0924 }                                                                       & \textcolor[rgb]{0.129,0.129,0.129}{0.9826}                                                                             & \textcolor[rgb]{0.129,0.129,0.129}{0.9736}                                                                       & 20                                                                                                              \\
\hline
ResNet                                              & \textcolor[rgb]{0.129,0.129,0.129}{0.2519 } & \textcolor[rgb]{0.129,0.129,0.129}{0.9063 } & \textcolor[rgb]{0.129,0.129,0.129}{0.2134 }                                                                       & \textcolor[rgb]{0.129,0.129,0.129}{0.9236}                                                                             & \textcolor[rgb]{0.129,0.129,0.129}{0.8805}                                                                       & 20                                                                                                              \\
\hline
ResNet-FT                                           & \textcolor[rgb]{0.129,0.129,0.129}{0.0102 } & \textcolor[rgb]{0.129,0.129,0.129}{0.9952 } & \textcolor[rgb]{0.129,0.129,0.129}{0.5594 }                                                                       & \textcolor[rgb]{0.129,0.129,0.129}{0.8576}                                                                             & \textcolor[rgb]{0.129,0.129,0.129}{0.8736}                                                                       & 20                                                                                                              \\
\hline
VGG16                                               & \textcolor[rgb]{0.129,0.129,0.129}{0.1057 } & \textcolor[rgb]{0.129,0.129,0.129}{0.9626 } & \textcolor[rgb]{0.129,0.129,0.129}{0.1063 }                                                                       & \textcolor[rgb]{0.129,0.129,0.129}{0.9618}                                                                             & \textcolor[rgb]{0.129,0.129,0.129}{0.9375}                                                                       & 20                                                                                                              \\
\hline
\rowcolor{green} VGG16-FT                           & \textcolor[rgb]{0.129,0.129,0.129}{1.099}   & \textcolor[rgb]{0.129,0.129,0.129}{0.3159 } & \textcolor[rgb]{0.129,0.129,0.129}{1.0977 }                                                                       & \textcolor[rgb]{0.129,0.129,0.129}{0.3576}                                                                             & \textcolor[rgb]{0.129,0.129,0.129}{0.3388}                                                                       & 20                                                                                                              \\
\hline
\end{tabular}
\end{table}

\begin{table}[htbp]
\begin{center}
\begin{tabular}{l|
>{\columncolor[HTML]{EFEFEF}}l |
>{\columncolor[HTML]{EFEFEF}}l |
>{\columncolor[HTML]{EFEFEF}}l |}
\hline
\multicolumn{1}{|l|}{\cellcolor[HTML]{C0C0C0}\textbf{V-COVID}}  & 0                                        & 235                                       & 0                                       \\ \hline
\multicolumn{1}{|l|}{\cellcolor[HTML]{C0C0C0}\textbf{V-Normal}} & 0                                        & 244                                       & 0                                       \\ \hline
\multicolumn{1}{|l|}{\cellcolor[HTML]{C0C0C0}\textbf{V-Neum.}}  & 0                                        & 241                                       & 0                                       \\ \hline
                                                                & \cellcolor[HTML]{C0C0C0}\textbf{P-COVID} & \cellcolor[HTML]{C0C0C0}\textbf{P-Normal} & \cellcolor[HTML]{C0C0C0}\textbf{P-Neum} \\ \cline{2-4} 
\end{tabular}
\end{center}
\end{table}

\subsection{InceptionV3}

\begin{figure}[H]
  \centering
  \begin{minipage}[b]{0.45\textwidth}
    \includegraphics[scale=0.75]{./images/inception1loss}
	\caption{Training-Validation Loss}
  \end{minipage}
  \hfill
  \begin{minipage}[b]{0.45\textwidth}
    \includegraphics[scale=0.75]{./images/inception1acc}
	\caption{Training-Validation accuracy}
  \end{minipage}
\end{figure}

\begin{table}[H]
\centering
\begin{tabular}{|c|c|c|c|c|c|c|}
\hline
\rowcolor[rgb]{0.753,0.753,0.753}  \textbf{Modelo}  & \textbf{Loss}                               & \textbf{Accuracy}                           & \begin{tabular}[c]{@{}>{\cellcolor[rgb]{0.753,0.753,0.753}}c@{}}\textbf{Validation}\\\textbf{ Loss} \end{tabular} & \begin{tabular}[c]{@{}>{\cellcolor[rgb]{0.753,0.753,0.753}}c@{}}\textbf{Validation }\\\textbf{ Accuracy} \end{tabular} & \begin{tabular}[c]{@{}>{\cellcolor[rgb]{0.753,0.753,0.753}}c@{}}\textbf{Test }\\\textbf{ Accuracy} \end{tabular} & \begin{tabular}[c]{@{}>{\cellcolor[rgb]{0.753,0.753,0.753}}c@{}}\textbf{Número}\\\textbf{Épocas} \end{tabular}  \\
\hline
\rowcolor[rgb]{0.937,0.937,0.937} DN-V0             &                                             &                                             &                                                                                                                   &                                                                                                                        & \textcolor[rgb]{0.129,0.129,0.129}{0.3264}                                                                       &                                                                                                                 \\
\hline
DN-V1                                               & \textcolor[rgb]{0.129,0.129,0.129}{0.6501}  & \textcolor[rgb]{0.129,0.129,0.129}{0.9588 } & \textcolor[rgb]{0.129,0.129,0.129}{3.1393}                                                                        & \textcolor[rgb]{0.129,0.129,0.129}{0.8333}                                                                             & \textcolor[rgb]{0.129,0.129,0.129}{0.8361}                                                                       & 20                                                                                                              \\
\hline
DN-V2                                               & \textcolor[rgb]{0.129,0.129,0.129}{0.0109 } & \textcolor[rgb]{0.129,0.129,0.129}{0.9973 } & \textcolor[rgb]{0.129,0.129,0.129}{0.1146 }                                                                       & \textcolor[rgb]{0.129,0.129,0.129}{0.9583}                                                                             & \textcolor[rgb]{0.129,0.129,0.129}{0.9597}                                                                       & 20                                                                                                              \\
\hline
DN-V2-reg                                           & \textcolor[rgb]{0.129,0.129,0.129}{0.0883 } & \textcolor[rgb]{0.129,0.129,0.129}{0.9683 } & \textcolor[rgb]{0.129,0.129,0.129}{0.1494 }                                                                       & \textcolor[rgb]{0.129,0.129,0.129}{0.9583}                                                                             & \textcolor[rgb]{0.129,0.129,0.129}{0.9569}                                                                       & 20                                                                                                              \\
\hline
DN-V3                                               & \textcolor[rgb]{0.129,0.129,0.129}{0.0089 } & \textcolor[rgb]{0.129,0.129,0.129}{0.9964 } & \textcolor[rgb]{0.129,0.129,0.129}{0.0924 }                                                                       & \textcolor[rgb]{0.129,0.129,0.129}{0.9826}                                                                             & \textcolor[rgb]{0.129,0.129,0.129}{0.9736}                                                                       & 20                                                                                                              \\
\hline
ResNet                                              & \textcolor[rgb]{0.129,0.129,0.129}{0.2519 } & \textcolor[rgb]{0.129,0.129,0.129}{0.9063 } & \textcolor[rgb]{0.129,0.129,0.129}{0.2134 }                                                                       & \textcolor[rgb]{0.129,0.129,0.129}{0.9236}                                                                             & \textcolor[rgb]{0.129,0.129,0.129}{0.8805}                                                                       & 20                                                                                                              \\
\hline
ResNet-FT                                           & \textcolor[rgb]{0.129,0.129,0.129}{0.0102 } & \textcolor[rgb]{0.129,0.129,0.129}{0.9952 } & \textcolor[rgb]{0.129,0.129,0.129}{0.5594 }                                                                       & \textcolor[rgb]{0.129,0.129,0.129}{0.8576}                                                                             & \textcolor[rgb]{0.129,0.129,0.129}{0.8736}                                                                       & 20                                                                                                              \\
\hline
VGG16                                               & \textcolor[rgb]{0.129,0.129,0.129}{0.1057 } & \textcolor[rgb]{0.129,0.129,0.129}{0.9626 } & \textcolor[rgb]{0.129,0.129,0.129}{0.1063 }                                                                       & \textcolor[rgb]{0.129,0.129,0.129}{0.9618}                                                                             & \textcolor[rgb]{0.129,0.129,0.129}{0.9375}                                                                       & 20                                                                                                              \\
\hline
VGG16-FT                                            & \textcolor[rgb]{0.129,0.129,0.129}{1.099}   & \textcolor[rgb]{0.129,0.129,0.129}{0.3159 } & \textcolor[rgb]{0.129,0.129,0.129}{1.0977 }                                                                       & \textcolor[rgb]{0.129,0.129,0.129}{0.3576}                                                                             & \textcolor[rgb]{0.129,0.129,0.129}{0.3388}                                                                       & 20                                                                                                              \\
\hline
\rowcolor{green} InceptionV3                        & \textcolor[rgb]{0.129,0.129,0.129}{0.2194 } & \textcolor[rgb]{0.129,0.129,0.129}{0.9231 } & \textcolor[rgb]{0.129,0.129,0.129}{0.2337 }                                                                       & \textcolor[rgb]{0.129,0.129,0.129}{0.9201}                                                                             & \textcolor[rgb]{0.129,0.129,0.129}{0.8903}                                                                       & 20                                                                                                              \\
\hline
\end{tabular}
\end{table}

\begin{table}[htbp]
\begin{center}
\begin{tabular}{l|
>{\columncolor[HTML]{EFEFEF}}l |
>{\columncolor[HTML]{EFEFEF}}l |
>{\columncolor[HTML]{EFEFEF}}l |}
\hline
\multicolumn{1}{|l|}{\cellcolor[HTML]{C0C0C0}\textbf{V-COVID}}  & 224                                      & 4                                         & 7                                       \\ \hline
\multicolumn{1}{|l|}{\cellcolor[HTML]{C0C0C0}\textbf{V-Normal}} & 17                                       & 210                                       & 17                                      \\ \hline
\multicolumn{1}{|l|}{\cellcolor[HTML]{C0C0C0}\textbf{V-Neum.}}  & 11                                       & 23                                        & 207                                     \\ \hline
                                                                & \cellcolor[HTML]{C0C0C0}\textbf{P-COVID} & \cellcolor[HTML]{C0C0C0}\textbf{P-Normal} & \cellcolor[HTML]{C0C0C0}\textbf{P-Neum} \\ \cline{2-4} 
\end{tabular}
\end{center}
\end{table}

\begin{figure}[H]
  \centering
  \begin{minipage}[b]{0.45\textwidth}
    \includegraphics[scale=0.75]{./images/inception2loss}
	\caption{Training-Validation Loss}
  \end{minipage}
  \hfill
  \begin{minipage}[b]{0.45\textwidth}
    \includegraphics[scale=0.75]{./images/inception2acc}
	\caption{Training-Validation accuracy}
  \end{minipage}
\end{figure}

\begin{table}[H]
\centering
\begin{tabular}{|c|c|c|c|c|c|c|}
\hline
\rowcolor[rgb]{0.753,0.753,0.753}  \textbf{Modelo}  & \textbf{Loss}                               & \textbf{Accuracy}                           & \begin{tabular}[c]{@{}>{\cellcolor[rgb]{0.753,0.753,0.753}}c@{}}\textbf{Validation}\\\textbf{ Loss} \end{tabular} & \begin{tabular}[c]{@{}>{\cellcolor[rgb]{0.753,0.753,0.753}}c@{}}\textbf{Validation }\\\textbf{ Accuracy} \end{tabular} & \begin{tabular}[c]{@{}>{\cellcolor[rgb]{0.753,0.753,0.753}}c@{}}\textbf{Test }\\\textbf{ Accuracy} \end{tabular} & \begin{tabular}[c]{@{}>{\cellcolor[rgb]{0.753,0.753,0.753}}c@{}}\textbf{Número}\\\textbf{Épocas} \end{tabular}  \\
\hline
\rowcolor[rgb]{0.937,0.937,0.937} DN-V0             &                                             &                                             &                                                                                                                   &                                                                                                                        & \textcolor[rgb]{0.129,0.129,0.129}{0.3264}                                                                       &                                                                                                                 \\
\hline
DN-V1                                               & \textcolor[rgb]{0.129,0.129,0.129}{0.6501}  & \textcolor[rgb]{0.129,0.129,0.129}{0.9588 } & \textcolor[rgb]{0.129,0.129,0.129}{3.1393}                                                                        & \textcolor[rgb]{0.129,0.129,0.129}{0.8333}                                                                             & \textcolor[rgb]{0.129,0.129,0.129}{0.8361}                                                                       & 20                                                                                                              \\
\hline
DN-V2                                               & \textcolor[rgb]{0.129,0.129,0.129}{0.0109 } & \textcolor[rgb]{0.129,0.129,0.129}{0.9973 } & \textcolor[rgb]{0.129,0.129,0.129}{0.1146 }                                                                       & \textcolor[rgb]{0.129,0.129,0.129}{0.9583}                                                                             & \textcolor[rgb]{0.129,0.129,0.129}{0.9597}                                                                       & 20                                                                                                              \\
\hline
DN-V2-reg                                           & \textcolor[rgb]{0.129,0.129,0.129}{0.0883 } & \textcolor[rgb]{0.129,0.129,0.129}{0.9683 } & \textcolor[rgb]{0.129,0.129,0.129}{0.1494 }                                                                       & \textcolor[rgb]{0.129,0.129,0.129}{0.9583}                                                                             & \textcolor[rgb]{0.129,0.129,0.129}{0.9569}                                                                       & 20                                                                                                              \\
\hline
DN-V3                                               & \textcolor[rgb]{0.129,0.129,0.129}{0.0089 } & \textcolor[rgb]{0.129,0.129,0.129}{0.9964 } & \textcolor[rgb]{0.129,0.129,0.129}{0.0924 }                                                                       & \textcolor[rgb]{0.129,0.129,0.129}{0.9826}                                                                             & \textcolor[rgb]{0.129,0.129,0.129}{0.9736}                                                                       & 20                                                                                                              \\
\hline
ResNet                                              & \textcolor[rgb]{0.129,0.129,0.129}{0.2519 } & \textcolor[rgb]{0.129,0.129,0.129}{0.9063 } & \textcolor[rgb]{0.129,0.129,0.129}{0.2134 }                                                                       & \textcolor[rgb]{0.129,0.129,0.129}{0.9236}                                                                             & \textcolor[rgb]{0.129,0.129,0.129}{0.8805}                                                                       & 20                                                                                                              \\
\hline
ResNet-FT                                           & \textcolor[rgb]{0.129,0.129,0.129}{0.0102 } & \textcolor[rgb]{0.129,0.129,0.129}{0.9952 } & \textcolor[rgb]{0.129,0.129,0.129}{0.5594 }                                                                       & \textcolor[rgb]{0.129,0.129,0.129}{0.8576}                                                                             & \textcolor[rgb]{0.129,0.129,0.129}{0.8736}                                                                       & 20                                                                                                              \\
\hline
VGG16                                               & \textcolor[rgb]{0.129,0.129,0.129}{0.1057 } & \textcolor[rgb]{0.129,0.129,0.129}{0.9626 } & \textcolor[rgb]{0.129,0.129,0.129}{0.1063 }                                                                       & \textcolor[rgb]{0.129,0.129,0.129}{0.9618}                                                                             & \textcolor[rgb]{0.129,0.129,0.129}{0.9375}                                                                       & 20                                                                                                              \\
\hline
VGG16-FT                                            & \textcolor[rgb]{0.129,0.129,0.129}{1.099}   & \textcolor[rgb]{0.129,0.129,0.129}{0.3159 } & \textcolor[rgb]{0.129,0.129,0.129}{1.0977 }                                                                       & \textcolor[rgb]{0.129,0.129,0.129}{0.3576}                                                                             & \textcolor[rgb]{0.129,0.129,0.129}{0.3388}                                                                       & 20                                                                                                              \\
\hline
InceptionV3                                         & \textcolor[rgb]{0.129,0.129,0.129}{0.2194 } & \textcolor[rgb]{0.129,0.129,0.129}{0.9231 } & \textcolor[rgb]{0.129,0.129,0.129}{0.2337 }                                                                       & \textcolor[rgb]{0.129,0.129,0.129}{0.9201}                                                                             & \textcolor[rgb]{0.129,0.129,0.129}{0.8903}                                                                       & 20                                                                                                              \\
\hline
\rowcolor{green} InceptionV3-FT                                      & \textcolor[rgb]{0.129,0.129,0.129}{0.0146 } & \textcolor[rgb]{0.129,0.129,0.129}{0.9953 } & \textcolor[rgb]{0.129,0.129,0.129}{0.1142 }                                                                       & \textcolor[rgb]{0.129,0.129,0.129}{0.9688}                                                                             & \textcolor[rgb]{0.129,0.129,0.129}{0.9792}                                                                       & 20                                                                                                              \\
\hline

\end{tabular}
\end{table}

\begin{table}[htbp]
\begin{center}
\begin{tabular}{l|
>{\columncolor[HTML]{EFEFEF}}l |
>{\columncolor[HTML]{EFEFEF}}l |
>{\columncolor[HTML]{EFEFEF}}l |}
\hline
\multicolumn{1}{|l|}{\cellcolor[HTML]{C0C0C0}\textbf{V-COVID}}  & 233                                      & 1                                         & 1                                       \\ \hline
\multicolumn{1}{|l|}{\cellcolor[HTML]{C0C0C0}\textbf{V-Normal}} & 0                                        & 243                                       & 1                                       \\ \hline
\multicolumn{1}{|l|}{\cellcolor[HTML]{C0C0C0}\textbf{V-Neum.}}  & 1                                        & 11                                        & 229                                     \\ \hline
                                                                & \cellcolor[HTML]{C0C0C0}\textbf{P-COVID} & \cellcolor[HTML]{C0C0C0}\textbf{P-Normal} & \cellcolor[HTML]{C0C0C0}\textbf{P-Neum} \\ \cline{2-4} 
\end{tabular}
\end{center}
\end{table}

\subsection{Xception}


\begin{figure}[H]
  \centering
  \begin{minipage}[b]{0.45\textwidth}
    \includegraphics[scale=0.75]{./images/xception1loss}
	\caption{Training-Validation Loss}
  \end{minipage}
  \hfill
  \begin{minipage}[b]{0.45\textwidth}
    \includegraphics[scale=0.75]{./images/xception1acc}
	\caption{Training-Validation accuracy}
  \end{minipage}
\end{figure}

\begin{table}[H]
\centering
\begin{tabular}{|c|c|c|c|c|c|c|}
\hline
\rowcolor[rgb]{0.753,0.753,0.753}  \textbf{Modelo}  & \textbf{Loss}                               & \textbf{Accuracy}                           & \begin{tabular}[c]{@{}>{\cellcolor[rgb]{0.753,0.753,0.753}}c@{}}\textbf{Validation}\\\textbf{ Loss} \end{tabular} & \begin{tabular}[c]{@{}>{\cellcolor[rgb]{0.753,0.753,0.753}}c@{}}\textbf{Validation }\\\textbf{ Accuracy} \end{tabular} & \begin{tabular}[c]{@{}>{\cellcolor[rgb]{0.753,0.753,0.753}}c@{}}\textbf{Test }\\\textbf{ Accuracy} \end{tabular} & \begin{tabular}[c]{@{}>{\cellcolor[rgb]{0.753,0.753,0.753}}c@{}}\textbf{Número}\\\textbf{Épocas} \end{tabular}  \\
\hline
\rowcolor[rgb]{0.937,0.937,0.937} DN-V0             &                                             &                                             &                                                                                                                   &                                                                                                                        & \textcolor[rgb]{0.129,0.129,0.129}{0.3264}                                                                       &                                                                                                                 \\
\hline
DN-V1                                               & \textcolor[rgb]{0.129,0.129,0.129}{0.6501}  & \textcolor[rgb]{0.129,0.129,0.129}{0.9588 } & \textcolor[rgb]{0.129,0.129,0.129}{3.1393}                                                                        & \textcolor[rgb]{0.129,0.129,0.129}{0.8333}                                                                             & \textcolor[rgb]{0.129,0.129,0.129}{0.8361}                                                                       & 20                                                                                                              \\
\hline
DN-V2                                               & \textcolor[rgb]{0.129,0.129,0.129}{0.0109 } & \textcolor[rgb]{0.129,0.129,0.129}{0.9973 } & \textcolor[rgb]{0.129,0.129,0.129}{0.1146 }                                                                       & \textcolor[rgb]{0.129,0.129,0.129}{0.9583}                                                                             & \textcolor[rgb]{0.129,0.129,0.129}{0.9597}                                                                       & 20                                                                                                              \\
\hline
DN-V2-reg                                           & \textcolor[rgb]{0.129,0.129,0.129}{0.0883 } & \textcolor[rgb]{0.129,0.129,0.129}{0.9683 } & \textcolor[rgb]{0.129,0.129,0.129}{0.1494 }                                                                       & \textcolor[rgb]{0.129,0.129,0.129}{0.9583}                                                                             & \textcolor[rgb]{0.129,0.129,0.129}{0.9569}                                                                       & 20                                                                                                              \\
\hline
DN-V3                                               & \textcolor[rgb]{0.129,0.129,0.129}{0.0089 } & \textcolor[rgb]{0.129,0.129,0.129}{0.9964 } & \textcolor[rgb]{0.129,0.129,0.129}{0.0924 }                                                                       & \textcolor[rgb]{0.129,0.129,0.129}{0.9826}                                                                             & \textcolor[rgb]{0.129,0.129,0.129}{0.9736}                                                                       & 20                                                                                                              \\
\hline
ResNet                                              & \textcolor[rgb]{0.129,0.129,0.129}{0.2519 } & \textcolor[rgb]{0.129,0.129,0.129}{0.9063 } & \textcolor[rgb]{0.129,0.129,0.129}{0.2134 }                                                                       & \textcolor[rgb]{0.129,0.129,0.129}{0.9236}                                                                             & \textcolor[rgb]{0.129,0.129,0.129}{0.8805}                                                                       & 20                                                                                                              \\
\hline
ResNet-FT                                           & \textcolor[rgb]{0.129,0.129,0.129}{0.0102 } & \textcolor[rgb]{0.129,0.129,0.129}{0.9952 } & \textcolor[rgb]{0.129,0.129,0.129}{0.5594 }                                                                       & \textcolor[rgb]{0.129,0.129,0.129}{0.8576}                                                                             & \textcolor[rgb]{0.129,0.129,0.129}{0.8736}                                                                       & 20                                                                                                              \\
\hline
VGG16                                               & \textcolor[rgb]{0.129,0.129,0.129}{0.1057 } & \textcolor[rgb]{0.129,0.129,0.129}{0.9626 } & \textcolor[rgb]{0.129,0.129,0.129}{0.1063 }                                                                       & \textcolor[rgb]{0.129,0.129,0.129}{0.9618}                                                                             & \textcolor[rgb]{0.129,0.129,0.129}{0.9375}                                                                       & 20                                                                                                              \\
\hline
VGG16-FT                                            & \textcolor[rgb]{0.129,0.129,0.129}{1.099}   & \textcolor[rgb]{0.129,0.129,0.129}{0.3159 } & \textcolor[rgb]{0.129,0.129,0.129}{1.0977 }                                                                       & \textcolor[rgb]{0.129,0.129,0.129}{0.3576}                                                                             & \textcolor[rgb]{0.129,0.129,0.129}{0.3388}                                                                       & 20                                                                                                              \\
\hline
InceptionV3                                         & \textcolor[rgb]{0.129,0.129,0.129}{0.2194 } & \textcolor[rgb]{0.129,0.129,0.129}{0.9231 } & \textcolor[rgb]{0.129,0.129,0.129}{0.2337 }                                                                       & \textcolor[rgb]{0.129,0.129,0.129}{0.9201}                                                                             & \textcolor[rgb]{0.129,0.129,0.129}{0.8903}                                                                       & 20                                                                                                              \\
\hline
InceptionV3-FT                                      & \textcolor[rgb]{0.129,0.129,0.129}{0.0146 } & \textcolor[rgb]{0.129,0.129,0.129}{0.9953 } & \textcolor[rgb]{0.129,0.129,0.129}{0.1142 }                                                                       & \textcolor[rgb]{0.129,0.129,0.129}{0.9688}                                                                             & \textcolor[rgb]{0.129,0.129,0.129}{0.9792}                                                                       & 20                                                                                                              \\
\hline
\rowcolor{green} Xception                           & \textcolor[rgb]{0.129,0.129,0.129}{0.1746 } & \textcolor[rgb]{0.129,0.129,0.129}{0.9405 } & \textcolor[rgb]{0.129,0.129,0.129}{0.1319 }                                                                       & \textcolor[rgb]{0.129,0.129,0.129}{0.9618}                                                                             & \textcolor[rgb]{0.129,0.129,0.129}{0.9153}                                                                       & 20                                                                                                              \\
\hline
\end{tabular}
\end{table}

\begin{table}[htbp]
\begin{center}
\begin{tabular}{l|
>{\columncolor[HTML]{EFEFEF}}l |
>{\columncolor[HTML]{EFEFEF}}l |
>{\columncolor[HTML]{EFEFEF}}l |}
\hline
\multicolumn{1}{|l|}{\cellcolor[HTML]{C0C0C0}\textbf{V-COVID}}  & 231                                      & 2                                         & 2                                       \\ \hline
\multicolumn{1}{|l|}{\cellcolor[HTML]{C0C0C0}\textbf{V-Normal}} & 2                                        & 223                                       & 19                                      \\ \hline
\multicolumn{1}{|l|}{\cellcolor[HTML]{C0C0C0}\textbf{V-Neum.}}  & 8                                        & 28                                        & 205                                     \\ \hline
                                                                & \cellcolor[HTML]{C0C0C0}\textbf{P-COVID} & \cellcolor[HTML]{C0C0C0}\textbf{P-Normal} & \cellcolor[HTML]{C0C0C0}\textbf{P-Neum} \\ \cline{2-4} 
\end{tabular}
\end{center}
\end{table}

\begin{figure}[H]
  \centering
  \begin{minipage}[b]{0.45\textwidth}
    \includegraphics[scale=0.75]{./images/xception2aloss}
	\caption{Training-Validation Loss}
  \end{minipage}
  \hfill
  \begin{minipage}[b]{0.45\textwidth}
    \includegraphics[scale=0.75]{./images/xception2acc}
	\caption{Training-Validation accuracy}
  \end{minipage}
\end{figure}

\begin{table}[H]
\centering
\begin{tabular}{|c|c|c|c|c|c|c|}
\hline
\rowcolor[rgb]{0.753,0.753,0.753}  \textbf{Modelo}  & \textbf{Loss}                               & \textbf{Accuracy}                           & \begin{tabular}[c]{@{}>{\cellcolor[rgb]{0.753,0.753,0.753}}c@{}}\textbf{Validation}\\\textbf{ Loss} \end{tabular} & \begin{tabular}[c]{@{}>{\cellcolor[rgb]{0.753,0.753,0.753}}c@{}}\textbf{Validation }\\\textbf{ Accuracy} \end{tabular} & \begin{tabular}[c]{@{}>{\cellcolor[rgb]{0.753,0.753,0.753}}c@{}}\textbf{Test }\\\textbf{ Accuracy} \end{tabular} & \begin{tabular}[c]{@{}>{\cellcolor[rgb]{0.753,0.753,0.753}}c@{}}\textbf{Número}\\\textbf{Épocas} \end{tabular}  \\
\hline
\rowcolor[rgb]{0.937,0.937,0.937} DN-V0             &                                             &                                             &                                                                                                                   &                                                                                                                        & \textcolor[rgb]{0.129,0.129,0.129}{0.3264}                                                                       &                                                                                                                 \\
\hline
DN-V1                                               & \textcolor[rgb]{0.129,0.129,0.129}{0.6501}  & \textcolor[rgb]{0.129,0.129,0.129}{0.9588 } & \textcolor[rgb]{0.129,0.129,0.129}{3.1393}                                                                        & \textcolor[rgb]{0.129,0.129,0.129}{0.8333}                                                                             & \textcolor[rgb]{0.129,0.129,0.129}{0.8361}                                                                       & 20                                                                                                              \\
\hline
DN-V2                                               & \textcolor[rgb]{0.129,0.129,0.129}{0.0109 } & \textcolor[rgb]{0.129,0.129,0.129}{0.9973 } & \textcolor[rgb]{0.129,0.129,0.129}{0.1146 }                                                                       & \textcolor[rgb]{0.129,0.129,0.129}{0.9583}                                                                             & \textcolor[rgb]{0.129,0.129,0.129}{0.9597}                                                                       & 20                                                                                                              \\
\hline
DN-V2-reg                                           & \textcolor[rgb]{0.129,0.129,0.129}{0.0883 } & \textcolor[rgb]{0.129,0.129,0.129}{0.9683 } & \textcolor[rgb]{0.129,0.129,0.129}{0.1494 }                                                                       & \textcolor[rgb]{0.129,0.129,0.129}{0.9583}                                                                             & \textcolor[rgb]{0.129,0.129,0.129}{0.9569}                                                                       & 20                                                                                                              \\
\hline
DN-V3                                               & \textcolor[rgb]{0.129,0.129,0.129}{0.0089 } & \textcolor[rgb]{0.129,0.129,0.129}{0.9964 } & \textcolor[rgb]{0.129,0.129,0.129}{0.0924 }                                                                       & \textcolor[rgb]{0.129,0.129,0.129}{0.9826}                                                                             & \textcolor[rgb]{0.129,0.129,0.129}{0.9736}                                                                       & 20                                                                                                              \\
\hline
ResNet                                              & \textcolor[rgb]{0.129,0.129,0.129}{0.2519 } & \textcolor[rgb]{0.129,0.129,0.129}{0.9063 } & \textcolor[rgb]{0.129,0.129,0.129}{0.2134 }                                                                       & \textcolor[rgb]{0.129,0.129,0.129}{0.9236}                                                                             & \textcolor[rgb]{0.129,0.129,0.129}{0.8805}                                                                       & 20                                                                                                              \\
\hline
ResNet-FT                                           & \textcolor[rgb]{0.129,0.129,0.129}{0.0102 } & \textcolor[rgb]{0.129,0.129,0.129}{0.9952 } & \textcolor[rgb]{0.129,0.129,0.129}{0.5594 }                                                                       & \textcolor[rgb]{0.129,0.129,0.129}{0.8576}                                                                             & \textcolor[rgb]{0.129,0.129,0.129}{0.8736}                                                                       & 20                                                                                                              \\
\hline
VGG16                                               & \textcolor[rgb]{0.129,0.129,0.129}{0.1057 } & \textcolor[rgb]{0.129,0.129,0.129}{0.9626 } & \textcolor[rgb]{0.129,0.129,0.129}{0.1063 }                                                                       & \textcolor[rgb]{0.129,0.129,0.129}{0.9618}                                                                             & \textcolor[rgb]{0.129,0.129,0.129}{0.9375}                                                                       & 20                                                                                                              \\
\hline
VGG16-FT                                            & \textcolor[rgb]{0.129,0.129,0.129}{1.099}   & \textcolor[rgb]{0.129,0.129,0.129}{0.3159 } & \textcolor[rgb]{0.129,0.129,0.129}{1.0977 }                                                                       & \textcolor[rgb]{0.129,0.129,0.129}{0.3576}                                                                             & \textcolor[rgb]{0.129,0.129,0.129}{0.3388}                                                                       & 20                                                                                                              \\
\hline
InceptionV3                                         & \textcolor[rgb]{0.129,0.129,0.129}{0.2194 } & \textcolor[rgb]{0.129,0.129,0.129}{0.9231 } & \textcolor[rgb]{0.129,0.129,0.129}{0.2337 }                                                                       & \textcolor[rgb]{0.129,0.129,0.129}{0.9201}                                                                             & \textcolor[rgb]{0.129,0.129,0.129}{0.8903}                                                                       & 20                                                                                                              \\
\hline
InceptionV3-FT                                      & \textcolor[rgb]{0.129,0.129,0.129}{0.0146 } & \textcolor[rgb]{0.129,0.129,0.129}{0.9953 } & \textcolor[rgb]{0.129,0.129,0.129}{0.1142 }                                                                       & \textcolor[rgb]{0.129,0.129,0.129}{0.9688}                                                                             & \textcolor[rgb]{0.129,0.129,0.129}{0.9792}                                                                       & 20                                                                                                              \\
\hline
Xception                                            & \textcolor[rgb]{0.129,0.129,0.129}{0.1746 } & \textcolor[rgb]{0.129,0.129,0.129}{0.9405 } & \textcolor[rgb]{0.129,0.129,0.129}{0.1319 }                                                                       & \textcolor[rgb]{0.129,0.129,0.129}{0.9618}                                                                             & \textcolor[rgb]{0.129,0.129,0.129}{0.9153}                                                                       & 20                                                                                                              \\
\hline
\rowcolor{green} Xception-FT                        & \textcolor[rgb]{0.129,0.129,0.129}{0.0157 } & \textcolor[rgb]{0.129,0.129,0.129}{0.9946 } & \textcolor[rgb]{0.129,0.129,0.129}{0.0352 }                                                                       & \textcolor[rgb]{0.129,0.129,0.129}{0.9861}                                                                             & \textcolor[rgb]{0.129,0.129,0.129}{0.9875}                                                                       & 20                                                                                                              \\
\hline
\end{tabular}
\end{table}


\begin{table}[htbp]
\begin{center}
\begin{tabular}{l|
>{\columncolor[HTML]{EFEFEF}}l |
>{\columncolor[HTML]{EFEFEF}}l |
>{\columncolor[HTML]{EFEFEF}}l |}
\hline
\multicolumn{1}{|l|}{\cellcolor[HTML]{C0C0C0}\textbf{V-COVID}}  & 234                                      & 0                                         & 1                                       \\ \hline
\multicolumn{1}{|l|}{\cellcolor[HTML]{C0C0C0}\textbf{V-Normal}} & 0                                        & 244                                       & 0                                       \\ \hline
\multicolumn{1}{|l|}{\cellcolor[HTML]{C0C0C0}\textbf{V-Neum.}}  & 0                                        & 8                                         & 233                                     \\ \hline
                                                                & \cellcolor[HTML]{C0C0C0}\textbf{P-COVID} & \cellcolor[HTML]{C0C0C0}\textbf{P-Normal} & \cellcolor[HTML]{C0C0C0}\textbf{P-Neum} \\ \cline{2-4} 
\end{tabular}
\end{center}
\end{table}

\subsection{Red from scratch}

\begin{table}[H]
\centering
\begin{tabular}{|c|c|c|c|c|c|c|}
\hline
\rowcolor[rgb]{0.753,0.753,0.753}  \textbf{Modelo}  & \textbf{Loss}                               & \textbf{Accuracy}                           & \begin{tabular}[c]{@{}>{\cellcolor[rgb]{0.753,0.753,0.753}}c@{}}\textbf{Validation}\\\textbf{ Loss} \end{tabular} & \begin{tabular}[c]{@{}>{\cellcolor[rgb]{0.753,0.753,0.753}}c@{}}\textbf{Validation }\\\textbf{ Accuracy} \end{tabular} & \begin{tabular}[c]{@{}>{\cellcolor[rgb]{0.753,0.753,0.753}}c@{}}\textbf{Test }\\\textbf{ Accuracy} \end{tabular} & \begin{tabular}[c]{@{}>{\cellcolor[rgb]{0.753,0.753,0.753}}c@{}}\textbf{Número}\\\textbf{Épocas} \end{tabular}  \\
\hline
\rowcolor[rgb]{0.937,0.937,0.937} DN-V0             &                                             &                                             &                                                                                                                   &                                                                                                                        & \textcolor[rgb]{0.129,0.129,0.129}{0.3264}                                                                       &                                                                                                                 \\
\hline
DN-V1                                               & \textcolor[rgb]{0.129,0.129,0.129}{0.6501}  & \textcolor[rgb]{0.129,0.129,0.129}{0.9588 } & \textcolor[rgb]{0.129,0.129,0.129}{3.1393}                                                                        & \textcolor[rgb]{0.129,0.129,0.129}{0.8333}                                                                             & \textcolor[rgb]{0.129,0.129,0.129}{0.8361}                                                                       & 20                                                                                                              \\
\hline
DN-V2                                               & \textcolor[rgb]{0.129,0.129,0.129}{0.0109 } & \textcolor[rgb]{0.129,0.129,0.129}{0.9973 } & \textcolor[rgb]{0.129,0.129,0.129}{0.1146 }                                                                       & \textcolor[rgb]{0.129,0.129,0.129}{0.9583}                                                                             & \textcolor[rgb]{0.129,0.129,0.129}{0.9597}                                                                       & 20                                                                                                              \\
\hline
DN-V2-reg                                           & \textcolor[rgb]{0.129,0.129,0.129}{0.0883 } & \textcolor[rgb]{0.129,0.129,0.129}{0.9683 } & \textcolor[rgb]{0.129,0.129,0.129}{0.1494 }                                                                       & \textcolor[rgb]{0.129,0.129,0.129}{0.9583}                                                                             & \textcolor[rgb]{0.129,0.129,0.129}{0.9569}                                                                       & 20                                                                                                              \\
\hline
DN-V3                                               & \textcolor[rgb]{0.129,0.129,0.129}{0.0089 } & \textcolor[rgb]{0.129,0.129,0.129}{0.9964 } & \textcolor[rgb]{0.129,0.129,0.129}{0.0924 }                                                                       & \textcolor[rgb]{0.129,0.129,0.129}{0.9826}                                                                             & \textcolor[rgb]{0.129,0.129,0.129}{0.9736}                                                                       & 20                                                                                                              \\
\hline
ResNet                                              & \textcolor[rgb]{0.129,0.129,0.129}{0.2519 } & \textcolor[rgb]{0.129,0.129,0.129}{0.9063 } & \textcolor[rgb]{0.129,0.129,0.129}{0.2134 }                                                                       & \textcolor[rgb]{0.129,0.129,0.129}{0.9236}                                                                             & \textcolor[rgb]{0.129,0.129,0.129}{0.8805}                                                                       & 20                                                                                                              \\
\hline
ResNet-FT                                           & \textcolor[rgb]{0.129,0.129,0.129}{0.0102 } & \textcolor[rgb]{0.129,0.129,0.129}{0.9952 } & \textcolor[rgb]{0.129,0.129,0.129}{0.5594 }                                                                       & \textcolor[rgb]{0.129,0.129,0.129}{0.8576}                                                                             & \textcolor[rgb]{0.129,0.129,0.129}{0.8736}                                                                       & 20                                                                                                              \\
\hline
VGG16                                               & \textcolor[rgb]{0.129,0.129,0.129}{0.1057 } & \textcolor[rgb]{0.129,0.129,0.129}{0.9626 } & \textcolor[rgb]{0.129,0.129,0.129}{0.1063 }                                                                       & \textcolor[rgb]{0.129,0.129,0.129}{0.9618}                                                                             & \textcolor[rgb]{0.129,0.129,0.129}{0.9375}                                                                       & 20                                                                                                              \\
\hline
VGG16-FT                                            & \textcolor[rgb]{0.129,0.129,0.129}{1.099}   & \textcolor[rgb]{0.129,0.129,0.129}{0.3159 } & \textcolor[rgb]{0.129,0.129,0.129}{1.0977 }                                                                       & \textcolor[rgb]{0.129,0.129,0.129}{0.3576}                                                                             & \textcolor[rgb]{0.129,0.129,0.129}{0.3388}                                                                       & 20                                                                                                              \\
\hline
InceptionV3                                         & \textcolor[rgb]{0.129,0.129,0.129}{0.2194 } & \textcolor[rgb]{0.129,0.129,0.129}{0.9231 } & \textcolor[rgb]{0.129,0.129,0.129}{0.2337 }                                                                       & \textcolor[rgb]{0.129,0.129,0.129}{0.9201}                                                                             & \textcolor[rgb]{0.129,0.129,0.129}{0.8903}                                                                       & 20                                                                                                              \\
\hline
InceptionV3-FT                                      & \textcolor[rgb]{0.129,0.129,0.129}{0.0146 } & \textcolor[rgb]{0.129,0.129,0.129}{0.9953 } & \textcolor[rgb]{0.129,0.129,0.129}{0.1142 }                                                                       & \textcolor[rgb]{0.129,0.129,0.129}{0.9688}                                                                             & \textcolor[rgb]{0.129,0.129,0.129}{0.9792}                                                                       & 20                                                                                                              \\
\hline
Xception                                            & \textcolor[rgb]{0.129,0.129,0.129}{0.1746 } & \textcolor[rgb]{0.129,0.129,0.129}{0.9405 } & \textcolor[rgb]{0.129,0.129,0.129}{0.1319 }                                                                       & \textcolor[rgb]{0.129,0.129,0.129}{0.9618}                                                                             & \textcolor[rgb]{0.129,0.129,0.129}{0.9153}                                                                       & 20                                                                                                              \\
\hline
Xception-FT                        & \textcolor[rgb]{0.129,0.129,0.129}{0.0157 } & \textcolor[rgb]{0.129,0.129,0.129}{0.9946 } & \textcolor[rgb]{0.129,0.129,0.129}{0.0352 }                                                                       & \textcolor[rgb]{0.129,0.129,0.129}{0.9861}                                                                             & \textcolor[rgb]{0.129,0.129,0.129}{0.9875}                                                                       & 20                                                                                                              \\
\hline
\rowcolor{green} CNN from scratch                        & \textcolor[rgb]{0.129,0.129,0.129}{1.0993  } & \textcolor[rgb]{0.129,0.129,0.129}{0.3233  } & \textcolor[rgb]{0.129,0.129,0.129}{1.0983 }                                                                       & \textcolor[rgb]{0.129,0.129,0.129}{0.3576}                                                                             & \textcolor[rgb]{0.129,0.129,0.129}{0.3388}                                                                       & 20                                                                                                              \\
\hline

\end{tabular}
\end{table}

\section{Análisis del modelo: mapas de activación, mapas de calor y conclusiones}

Un aspecto que hace que las redes neuronales sean utilizadas con mucha discreción en cualquier ámbito del conocimiento es porque son vistas como cajas negras las cuales no se sabe muy bien qué aprenden y sobre qué características concretas basan sus decisiones, con lo que es necesario que haya algún otro criterio que sea capaz la decisión que aporta una red neuronal, y en particular, también ocurre con las redes neuronales convolucionales. \\

Nuestro objetivo en esta sección es intentar comprender un poco mejor qué características aprenden nuestras redes y qué elementos de la imagen consideran más importantes a la hora de tomar su decisión.  La motivación que tenemos para hacerlo es debido a que en nuestras imágenes tenemos otros elementos que no pertenecen al aparato respiratorio, y que podrían estar afectando negativamente a las decisiones que toman nuestras redes. Puesto que además usamos redes preentrenadas, puede que parte de las características que se detecten debido a dicho preentrenamiento sean reconocidas en lugares de la imagen que no sean del aparato respiratorio. Podrían ser elementos óseos o, de forma más problemática, letras que aparecen en algunas de las imágenes de nuestro conjunto de datos, las cuales son añadidas por algún programa tras haber obtenido la imagen radiográfica inicial.  Si la decisión que toma nuestra red estuviera basada en gran medida en dichas características, independientemente de los buenos resultados que diera esa red, sería indeseable y deberíamos descartarlos, puesto que estaría sobreajustando al conjunto de datos usado directamente, y encima usando características nada relevantes para la tarea que nos compete.


\subsection{Mapas de activación}

\subsection{Mapas de calor}

RESNET

\includegraphics[width=0.85\textwidth]{./images/resnetfilters}

\includegraphics[width=0.85\textwidth]{./images/resnetheatmap}

XCEPTION

\includegraphics[width=0.85\textwidth]{./images/xceptionfilters}

\includegraphics[width=0.85\textwidth]{./images/heatmapxception}



\subsection{Conclusiones}

\section{Posibles propuestas de mejora}

Como posibles propuestas para mejoras del modelo se podrían tratar los siguientes aspectos:

\begin{itemize}
\item Modificar diferentes conjuntos de hiperparámetros de la redes , como por ejemplo parámetros del data augmentation o dropout y seleccionar el conjunto para el que el conjunto de validación haya dado mejores resultados.  Puesto que hemos probado muchas redes, el tiempo de ejecución hubiera crecido bastante, pero reconocemos que todavía hay margen de mejora empleando el conjunto de validación para poder optimizar más los resultados de la red variando los hiperparámetros.
\item Se podrían incluir más técnicas de visualización, como por ejemplo calcular las top n patches para algunas de las neuronas de la red, con el objetivo de ver qué regiones de las imágenes han dado mayor valor de activación para esas neuronas. Seleccionando algunas neuronas situadas cerca de la capa de clasificación, cuyo campo receptivo sería prácticamente toda la red, podremos observar cuáles son las imágenes que más claramente dan signos de ser normales, tener covid, o tener otro tipo de neumonías víricas, cosa que sería muy interesante para establecer los ejemplos más característicos de cada conjunto. Además podríamos seleccionar neuronas más alejadas de las capas finales, que tengan un campo receptivo menor, para ver qué patrones de medio tamaño se reconocen en la red que luego serán útiles para la clasificación, ya que facilitaría mucho también el encontrar patrones en una imagen que fueran determinante a la hora de decidir la clase de una imagen.
\item Como hemos podido observar en los mapas de calor, cuando usamos una red preentrenada como extractor de características,  las redes neuronales focalizan su atención en muchas ocasiones en distintas zonas de la imagen que muy poco o nada tienen que ver con los pulmones, donde es evidente que es donde se puede deducir la presencia o no de enfermedad,  tal y como lo muestran los modelos de fine tuning que hemos podido visualizar por mapas de calor.  Lo que podríamos hacer es forzar a nuestro modelo a prestar atención a los pulmones en vez de a los bordes de la imagen, los cuales normalmente tienen letras u otros elementos perniciosos. Para ello podríamos intentar oscurecer los bordes de la imagen,  redimensionar la agrandándola y recortarla dejando fuera los bordes, o podríamos utilizar algún método más sofisticado donde pudiéramos reconocer y eliminar las letras, para posteriormente hacer una reconstrucción de la imagen.  Existen muchas alternativas para forzar a que la red preste atención a la zona pulmonar. Esto no va en contra de la filosofía de las redes neuronales, en cuanto a que ellas deciden qué patrones aprender a reconocer reconocer, puesto que lo único que hacemos es focalizar su atención en la zona de la imagen que nosotros queremos.
\end{itemize}


\section{Referencias}

https://www.kaggle.com/tawsifurrahman/covid19-radiography-database


\section{Referencias}

https://www.kaggle.com/tawsifurrahman/covid19-radiography-database






\end{document}
